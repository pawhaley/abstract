%%%%%%%%%%%%%%%%%%%%%%%%%%%%%%%%%%%%%%%%%%%%%%%%%%%%%%%%%%%%%%%%%%%%%%%%%%%%%%%%%%%%%%%
%%%%%%%%%%%%%%%%%%%%%%%%%%%%%%%%%%%%%%%%%%%%%%%%%%%%%%%%%%%%%%%%%%%%%%%%%%%%%%%%%%%%%%%
% 
% This top part of the document is called the 'preamble'.  Modify it with caution!
%
% The real document starts below where it says 'The main document starts here'.

\documentclass[12pt]{article}

\usepackage{amssymb,amsmath,amsthm}
\usepackage[top=1in, bottom=1in, left=1.25in, right=1.25in]{geometry}
\usepackage{fancyhdr}
\usepackage{enumerate}
\usepackage[bw,framed,numbered]{mcode}
\usepackage{color}

% Comment the following line to use TeX's default font of Computer Modern.
\usepackage{times,txfonts}

\newtheoremstyle{homework}% name of the style to be used
  {18pt}% measure of space to leave above the theorem. E.g.: 3pt
  {12pt}% measure of space to leave below the theorem. E.g.: 3pt
  {}% name of font to use in the body of the theorem
  {}% measure of space to indent
  {\bfseries}% name of head font
  {:}% punctuation between head and body
  {2ex}% space after theorem head; " " = normal interword space
  {}% Manually specify head
\theoremstyle{homework} 

% Set up an Exercise environment and a Solution label.
\newtheorem*{exercisecore}{Exercise \@currentlabel}
\newenvironment{exercise}[1]
{\def\@currentlabel{#1}\exercisecore}
{\endexercisecore}

\newcommand\W{{\color{red}\textbf{(W) (Hand this one in to David.)}}}
\newcommand\tome{{\color{red}\textbf{(Hand this one in to David.)}}}

\newcommand{\localhead}[1]{\par\smallskip\noindent\textbf{#1}\nobreak\\}%
\newcommand\solution{\localhead{Solution:}}

%%%%%%%%%%%%%%%%%%%%%%%%%%%%%%%%%%%%%%%%%%%%%%%%%%%%%%%%%%%%%%%%%%%%%%%%
%
% Stuff for getting the name/document date/title across the header
\makeatletter
\RequirePackage{fancyhdr}
\pagestyle{fancy}
\fancyfoot[C]{\ifnum \value{page} > 1\relax\thepage\fi}
\fancyhead[L]{\ifx\@doclabel\@empty\else\@doclabel\fi}
\fancyhead[C]{\ifx\@docdate\@empty\else\@docdate\fi}
\fancyhead[R]{\ifx\@docauthor\@empty\else\@docauthor\fi}
\headheight 15pt

\def\doclabel#1{\gdef\@doclabel{#1}}
\doclabel{Use {\tt\textbackslash doclabel\{MY LABEL\}}.}
\def\docdate#1{\gdef\@docdate{#1}}
\docdate{Use {\tt\textbackslash docdate\{MY DATE\}}.}
\def\docauthor#1{\gdef\@docauthor{#1}}
\docauthor{Use {\tt\textbackslash docauthor\{MY NAME\}}.}
\makeatother

% Shortcuts for blackboard bold number sets (reals, integers, etc.)
\newcommand{\Reals}{\ensuremath{\mathbb R}}
\newcommand{\Nats}{\ensuremath{\mathbb N}}
\newcommand{\Ints}{\ensuremath{\mathbb Z}}
\newcommand{\Rats}{\ensuremath{\mathbb Q}}
\newcommand{\Cplx}{\ensuremath{\mathbb C}}
%% Some equivalents that some people may prefer.
\let\RR\Reals
\let\NN\Nats
\let\II\Ints
\let\CC\Cplx

%%%%%%%%%%%%%%%%%%%%%%%%%%%%%%%%%%%%%%%%%%%%%%%%%%%%%%%%%%%%%%%%%%%%%%%%%%%%%%%%%%%%%%%
%%%%%%%%%%%%%%%%%%%%%%%%%%%%%%%%%%%%%%%%%%%%%%%%%%%%%%%%%%%%%%%%%%%%%%%%%%%%%%%%%%%%%%%
% 
% The main document start here.

% The following commands set up the material that appears in the header.
\doclabel{Math 405: HW 5}
\docauthor{Parker Whaley}
\docdate{Due feb 9, 2017}
\begin{document}
\begin{exercise}
{4.13}
I started by looking at the elements of $<21>$ and $<10>$, there intersection is $\{18,12,6\}$ as shown below.  Next I tested $<18>=\{18,12,6\}$ so $18$ is a generator for the intersection of these two.
\lstinputlisting{d1.txt}
Examining the $<a^{21}>\cap <a^{10}>$ described in the book we note that this is identical to $Z_{24}$ with $a=1$, thus by symmetry we can say $<a^{21}>\cap <a^{10}>=<a^{18}>$ and more in general $<a^{m}>\cap <a^{n}>=<a^{i}>$ if and only if $<m>\cap <n>=<i>$.
\end{exercise}
\begin{exercise}
{4.15}
\begin{enumerate}
\item
Closure.\\
Suppose $a,b\in H$.  Note that there must exist some naturals $k,j$ such that $|a|k=12$ and $|b|j=12$.  Note that $(ab)^{12}=a^{12}b^{12}=a^{|a|k}b^{|b|j}=e$, thus $|ab|$ divides $12$, and so we have closure.\
\item
Suppose $a\in H$ and $b\in <a>$.  Note that there exists some natural $i$ such that $a^i=b$.  Note that $b^{12}=(a^i)^{12}=(a^{12})^i=e^i=e$, thus $|b|$ divides $12$.  We conclude $<a>\subseteq H$ and since $a^{-1}\in <a>$ we know $a^{-1}\in H$.\\\\
By the two step subgroup test we conclude $H$ is a subgroup of $G$.\\\\
Nowhere did I use any special properties of $12$ and thus this is completely expandable to replacing the $12$ with any natural number.
\end{enumerate}
\end{exercise}
\begin{exercise}
{4.20}
Suppose there is no element that generates the entire group.  Select $a\in G$ where $a\neq e$.  Note that $a^{35}=e$, thus $|a|\mid 35$.  Note that there are only two possibilities eater $|a|=5$ or $|a|=7$.\\\\
Suppose all non identity elements have order $5$.  Suppose $a\in G-\{e\}$ and $b\in G-<a>$.  Suppose there exists $c\in <a>\cap <b>-\{e\}$.  Note that since $5$ is prime $c$ must be a generator of both $<a>$ and $<b>$ thus $<b> = <a>$, a contradiction, we conclude that no such $c$ exists so, $<a>\cap <b>=\{e\}$.  Noting that there is no overlap between subgroup cycles other than the identity we can conclude that there are exactly $\frac{35-1}{4}$ cyclic sub groups, $-1$ for the identity and $/4$ since each sub group has $4$ non-identity elements that appear in no other cycle.  We note that $\frac{35-1}{4}$ is not a natural number, a contradiction with it being the exact number of cyclic sub groups, we conclude that not all non identity elements have order $5$.\\\\
A very similar proof holds for $7$, and since $\frac{35-1}{6}$ is also not a whole number we conclude that not all non identity elements have order $7$.\\\\
We can now conclude that there must be at least one element of order $5$ and one element of order $7$.  Let $|a|=5$ and $|b|=7$.  Note that $ab\in G$.  Suppose $|ab|=5$.  Note that $e=(ab)^5=a^5b^5=eb^5=b^5$ and so $|b|=5$ a contradiction.  The same follows for $7$ and thus we conclude $|ab|\neq 5$ and $|ab|\neq 7$, this is impossible since $5$ and $7$ are the only possibilities for $|ab|$.  We conclude the negation of our first supposition and conclude that there is a element that generates the entire group, thus $G$ is cyclic.\\\\
The same would hold for $33$ as well as any other semi-prime.
\end{exercise}
\begin{exercise}
{4.33}
See attached
\end{exercise}
\begin{exercise}
{4.41}
Note that $ab$ is a group element thus $(ab)^{|ab|}=a^{|ab|}b^{|ab|}=e$.  Since $b^{|ab|}=a^{-|ab|}\in <a>$ and $b^{|ab|}\in <b>$ we can conclude $b^{|ab|}=e$ and $a^{|ab|}=e$, this means $|ab|$ is a multiple of $|a|=m$ and $|b|=n$, or $\text{lcm}(m,n)\leq |ab|$.  Let $n=\text{lcm}(m,n)$.  Note that $(ab)^{n}=a^{n}b^{n}=ee=e$, thus $|ab|\leq \text{lcm}(m,n)$.  We are forced to conclude that $ab$ has order of the least common multiple of $m$ and $n$.\\\\
As a counter example to prove that it is necessary that $a$ and $b$ commute, consider $D_3$, the symmetries of a triangle.  The rotations are order $3$ and the reflections are order $2$.  The only shared element between the cyclic group containing the first rotation and the cyclic group containing the first reflection is the identity.  The LCM of the group containing the rotation and the group containing the reflection is $6$ however there are no elements with order $6$.
\end{exercise}
\begin{exercise}
{4.64}
Suppose $c\in <a>\cap <b>-\{e\}$.  Note that $|c|\mid |a|$ and $|c|\mid |b|$, a contradiction since $1<|c|$, and $|a|$ and $|b|$ are relatively prime.
\end{exercise}
\end{document}