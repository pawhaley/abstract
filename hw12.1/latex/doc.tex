\documentclass[12pt]{article}

\usepackage{amssymb,amsmath,amsthm}
\usepackage[top=1in, bottom=1in, left=1.25in, right=1.25in]{geometry}
\usepackage{fancyhdr}
\usepackage{enumerate}
\usepackage[bw,framed,numbered]{mcode}
\usepackage{graphicx}

% Comment the following line to use TeX's default font of Computer Modern.
\usepackage{times,txfonts}

\newtheoremstyle{homework}% name of the style to be used
  {18pt}% measure of space to leave above the theorem. E.g.: 3pt
  {12pt}% measure of space to leave below the theorem. E.g.: 3pt
  {}% name of font to use in the body of the theorem
  {}% measure of space to indent
  {\bfseries}% name of head font
  {:}% punctuation between head and body
  {2ex}% space after theorem head; " " = normal interword space
  {}% Manually specify head
\theoremstyle{homework} 

% Set up an Exercise environment and a Solution label.
\newtheorem*{exercisecore}{Exercise \@currentlabel}
\newenvironment{exercise}[1]
{\def\@currentlabel{#1}\exercisecore}
{\endexercisecore}

\newcommand{\localhead}[1]{\par\smallskip\noindent\textbf{#1}\nobreak\\}%
\newcommand\solution{\localhead{Solution:}}

%%%%%%%%%%%%%%%%%%%%%%%%%%%%%%%%%%%%%%%%%%%%%%%%%%%%%%%%%%%%%%%%%%%%%%%%
%
% Stuff for getting the name/document date/title across the header
\makeatletter
\RequirePackage{fancyhdr}
\pagestyle{fancy}
\fancyfoot[C]{\ifnum \value{page} > 1\relax\thepage\fi}
\fancyhead[L]{\ifx\@doclabel\@empty\else\@doclabel\fi}
\fancyhead[C]{\ifx\@docdate\@empty\else\@docdate\fi}
\fancyhead[R]{\ifx\@docauthor\@empty\else\@docauthor\fi}
\headheight 15pt

\def\doclabel#1{\gdef\@doclabel{#1}}
\doclabel{Use {\tt\textbackslash doclabel\{MY LABEL\}}.}
\def\docdate#1{\gdef\@docdate{#1}}
\docdate{Use {\tt\textbackslash docdate\{MY DATE\}}.}
\def\docauthor#1{\gdef\@docauthor{#1}}
\docauthor{Use {\tt\textbackslash docauthor\{MY NAME\}}.}
\makeatother

% Shortcuts for blackboard bold number sets (reals, integers, etc.)
\newcommand{\Reals}{\ensuremath{\mathbb R}}
\newcommand{\Nats}{\ensuremath{\mathbb N}}
\newcommand{\Ints}{\ensuremath{\mathbb Z}}
\newcommand{\Rats}{\ensuremath{\mathbb Q}}
\newcommand{\Cplx}{\ensuremath{\mathbb C}}
\newcommand{\Aut}{\ensuremath{\text{Aut}}}
%% Some equivalents that some people may prefer.
\let\RR\Reals
\let\NN\Nats
\let\II\Ints
\let\CC\Cplx

%%%%%%%%%%%%%%%%%%%%%%%%%%%%%%%%%%%%%%%%%%%%%%%%%%%%%%%%%%%%%%%%%%%%%%%%%%%%%%%%%%%%%%%
%%%%%%%%%%%%%%%%%%%%%%%%%%%%%%%%%%%%%%%%%%%%%%%%%%%%%%%%%%%%%%%%%%%%%%%%%%%%%%%%%%%%%%%
% 
% The main document start here.

% The following commands set up the material that appears in the header.

%%%%%%%%%%%%%%%%%%%%%%%%%%%%%%%%%%%%%%%%%%%%%%%%%%%%%%%%%%%%%%%%%%%%%%%%%%%%%%%%%%%%%%%
%%%%%%%%%%%%%%%%%%%%%%%%%%%%%%%%%%%%%%%%%%%%%%%%%%%%%%%%%%%%%%%%%%%%%%%%%%%%%%%%%%%%%%%
% 
% The main document start here.

% The following commands set up the material that appears in the header.
\doclabel{Abstract hw 12-1}
\docauthor{Parker Whaley}
\docdate{Apr 6 2017}
%\lstinputlisting{}
\newcommand{\vv}{\mathbf{v}}
\begin{document}
\begin{exercise}{11.6}
Show that there are two Abelian groups of order 108 that have exactly one subgroup of order 3.\\
Note that $108=2^23^3$.  Thus any Abelian group of order 108 is isomorphic to one of the following:\\
\begin{tabular}{|l|l|l|}
\hline
group & exactly one subgroup of order 3  & and it is\\
\hline
$Z_{3^3}\bigoplus Z_{2^2}$ & yes & $<(9,0)>$\\
\hline
$Z_{3^3}\bigoplus Z_{2}\bigoplus Z_2$ & yes & $<(9,0,0)>$\\
\hline
$Z_{3^2}\bigoplus Z_3\bigoplus Z_{2^2}$ & no &  \\
\hline
$Z_{3^2}\bigoplus Z_3\bigoplus Z_{2}\bigoplus Z_2$ & no &  \\
\hline
$Z_{3}\bigoplus Z_3\bigoplus Z_3\bigoplus Z_{2^2}$ & no &  \\
\hline
$Z_{3}\bigoplus Z_3\bigoplus Z_3\bigoplus Z_{2}\bigoplus Z_2$ & no &  \\
\hline
\end{tabular}
\end{exercise}

\begin{exercise}{11.7}
Show that there are two Abelian groups of order 108 that have exactly four subgroups of order 3.\\
Note that $108=2^23^3$.  Thus any Abelian group of order 108 is isomorphic to one of the following:\\
\begin{tabular}{|l|l|l|}
\hline
group & has?  & and they are\\
\hline
$Z_{3^3}\bigoplus Z_{2^2}$ & no & \\
\hline
$Z_{3^3}\bigoplus Z_{2}\bigoplus Z_2$ & no & \\
\hline
$Z_{3^2}\bigoplus Z_3\bigoplus Z_{2^2}$ & yes & $<(3,0,0)>,<(3,1,0)>,<(0,1,0)>,<(3,2,0)>$ \\
\hline
$Z_{3^2}\bigoplus Z_3\bigoplus Z_{2}\bigoplus Z_2$ & yes & the above with a additional 0 \\
\hline
$Z_{3}\bigoplus Z_3\bigoplus Z_3\bigoplus Z_{2^2}$ & no &  \\
\hline
$Z_{3}\bigoplus Z_3\bigoplus Z_3\bigoplus Z_{2}\bigoplus Z_2$ & no &  \\
\hline
\end{tabular}
\end{exercise}

\begin{exercise}{11.8}
Show that there are two Abelian groups of order 108 that have exactly 13 subgroups of order 3.\\
Note that $108=2^23^3$.  Thus any Abelian group of order 108 is isomorphic to one of the following:\\
\begin{tabular}{|l|l|l|}
\hline
group & has?  & and they are\\
\hline
$Z_{3^3}\bigoplus Z_{2^2}$ & no & \\
\hline
$Z_{3^3}\bigoplus Z_{2}\bigoplus Z_2$ & no & \\
\hline
$Z_{3^2}\bigoplus Z_3\bigoplus Z_{2^2}$ & no &  \\
\hline
$Z_{3^2}\bigoplus Z_3\bigoplus Z_{2}\bigoplus Z_2$ & no &  \\
\hline
$Z_{3}\bigoplus Z_3\bigoplus Z_3\bigoplus Z_{2^2}$ & yes & see below \\
\hline
$Z_{3}\bigoplus Z_3\bigoplus Z_3\bigoplus Z_{2}\bigoplus Z_2$ & yes & the above with a additional 0 \\
\hline
\end{tabular}


For the above section $<(1,0,0,0)>,<(0,1,0,0)>,<(0,0,1,0)>,<(1,1,0,0)>,<(1,2,0,0)>,<(1,0,1,0)>,<(1,0,2,0)>,<(0,1,1,0)>,<(0,1,2,0)>,<(1,1,1,0)>,$
$<(1,1,2,0)>,<(1,2,1,0)>,<(1,2,2,0)>$.
\end{exercise}

\begin{exercise}{11.10}
Find all Abelian groups (up to isomorphism) of order 360.\\
Note that $360=5\cdot 3^2\cdot 2^3$ thus all Abelian groups of order 360 are isomorphic to one of the following:\\
\begin{tabular}{|l|}
\hline
$Z_5\bigoplus Z_{3}\bigoplus Z_{3}\bigoplus Z_{2}\bigoplus Z_{2}\bigoplus Z_{2}$\\
\hline
$Z_5\bigoplus Z_{3}\bigoplus Z_{3}\bigoplus Z_{2^2}\bigoplus Z_{2}$\\
\hline
$Z_5\bigoplus Z_{3}\bigoplus Z_{3}\bigoplus Z_{2^3}$\\
\hline
$Z_5\bigoplus Z_{3^2}\bigoplus Z_{2}\bigoplus Z_{2}\bigoplus Z_{2}$\\
\hline
$Z_5\bigoplus Z_{3^2}\bigoplus Z_{2^2}\bigoplus Z_{2}$\\
\hline
$Z_5\bigoplus Z_{3^2}\bigoplus Z_{2^3}$\\
\hline
\end{tabular}

\end{exercise}

\begin{exercise}{11.12}
Suppose that the order of some finite Abelian group is divisible by 10. Prove that the group has a cyclic subgroup of order 10.\\
Note that there must exist some isomorphisim $\phi$ that maps our Abelian group to something of the form $Z_{(p_1)^{n_1}}\bigoplus Z_{(p_2)^{n_2}}\cdots$.  Note that the primes 2 and 5 must divide the order of the group and thus there are $p_k=5$ and $p_l=2$.  WLoG let $k=1$, $l=2$.  Note that the element $(5^{n_1-1},2^{n_2-1},0,0,0\cdots )$ is of order 10, thus if we move back by $\phi^{-1}$ we have found a element in our original group of order 10.
\end{exercise}


\end{document}