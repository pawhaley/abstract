\documentclass[12pt]{article}

\usepackage{amssymb,amsmath,amsthm}
\usepackage[top=1in, bottom=1in, left=1.25in, right=1.25in]{geometry}
\usepackage{fancyhdr}
\usepackage{enumerate}
\usepackage[bw,framed,numbered]{mcode}
\usepackage{graphicx}

% Comment the following line to use TeX's default font of Computer Modern.
\usepackage{times,txfonts}

\newtheoremstyle{homework}% name of the style to be used
  {18pt}% measure of space to leave above the theorem. E.g.: 3pt
  {12pt}% measure of space to leave below the theorem. E.g.: 3pt
  {}% name of font to use in the body of the theorem
  {}% measure of space to indent
  {\bfseries}% name of head font
  {:}% punctuation between head and body
  {2ex}% space after theorem head; " " = normal interword space
  {}% Manually specify head
\theoremstyle{homework} 

% Set up an Exercise environment and a Solution label.
\newtheorem*{exercisecore}{Exercise \@currentlabel}
\newenvironment{exercise}[1]
{\def\@currentlabel{#1}\exercisecore}
{\endexercisecore}

\newcommand{\localhead}[1]{\par\smallskip\noindent\textbf{#1}\nobreak\\}%
\newcommand\solution{\localhead{Solution:}}

%%%%%%%%%%%%%%%%%%%%%%%%%%%%%%%%%%%%%%%%%%%%%%%%%%%%%%%%%%%%%%%%%%%%%%%%
%
% Stuff for getting the name/document date/title across the header
\makeatletter
\RequirePackage{fancyhdr}
\pagestyle{fancy}
\fancyfoot[C]{\ifnum \value{page} > 1\relax\thepage\fi}
\fancyhead[L]{\ifx\@doclabel\@empty\else\@doclabel\fi}
\fancyhead[C]{\ifx\@docdate\@empty\else\@docdate\fi}
\fancyhead[R]{\ifx\@docauthor\@empty\else\@docauthor\fi}
\headheight 15pt

\def\doclabel#1{\gdef\@doclabel{#1}}
\doclabel{Use {\tt\textbackslash doclabel\{MY LABEL\}}.}
\def\docdate#1{\gdef\@docdate{#1}}
\docdate{Use {\tt\textbackslash docdate\{MY DATE\}}.}
\def\docauthor#1{\gdef\@docauthor{#1}}
\docauthor{Use {\tt\textbackslash docauthor\{MY NAME\}}.}
\makeatother

% Shortcuts for blackboard bold number sets (reals, integers, etc.)
\newcommand{\Reals}{\ensuremath{\mathbb R}}
\newcommand{\Nats}{\ensuremath{\mathbb N}}
\newcommand{\Ints}{\ensuremath{\mathbb Z}}
\newcommand{\Rats}{\ensuremath{\mathbb Q}}
\newcommand{\Cplx}{\ensuremath{\mathbb C}}
%% Some equivalents that some people may prefer.
\let\RR\Reals
\let\NN\Nats
\let\II\Ints
\let\CC\Cplx

%%%%%%%%%%%%%%%%%%%%%%%%%%%%%%%%%%%%%%%%%%%%%%%%%%%%%%%%%%%%%%%%%%%%%%%%%%%%%%%%%%%%%%%
%%%%%%%%%%%%%%%%%%%%%%%%%%%%%%%%%%%%%%%%%%%%%%%%%%%%%%%%%%%%%%%%%%%%%%%%%%%%%%%%%%%%%%%
% 
% The main document start here.

% The following commands set up the material that appears in the header.

%%%%%%%%%%%%%%%%%%%%%%%%%%%%%%%%%%%%%%%%%%%%%%%%%%%%%%%%%%%%%%%%%%%%%%%%%%%%%%%%%%%%%%%
%%%%%%%%%%%%%%%%%%%%%%%%%%%%%%%%%%%%%%%%%%%%%%%%%%%%%%%%%%%%%%%%%%%%%%%%%%%%%%%%%%%%%%%
% 
% The main document start here.

% The following commands set up the material that appears in the header.
\doclabel{STAT PHYS: Homework 1}
\docauthor{Parker Whaley}
\docdate{Feb 8, 2017}

\newcommand{\vv}{\mathbf{v}}
\begin{document}
\begin{exercise}{5.11}
Determine whether the following permutations are even or odd.
\begin{enumerate}[(a)]
\item[(d)]
(12)(134)(152)\\
This would be made of 5 two cycles and thus is odd.
\item[e]
(1243)(3521)\\
This would be made of 6 two cycles and thus is even.
\end{enumerate}
\end{exercise}

\begin{exercise}{5.14}
Find eight elements in $S_6$ that commute with $(12)(34)(56)$. Do they form a subgroup of $S_6$ ?\\
Eight elements that commute are $A=\{e,(12),(12)(34),(12)(56),(12)(34)(56),(34),(34)(56),(56)\}\subseteq S_6$.  Yes they do form a sub group.  Note that every element is its own inverse.  Define $D_1=(12)$, $D_2=(34)$, $D_3=(56)$.  Note that each of these $D$s commute and are there own inverses.  Note that every combination of $D$s appear in $A$.  Take two arbitrary elements of $A$, $D_1^aD_2^bD_3^c$ and $D_1^dD_2^eD_3^f$, note that combining them $D_1^{ad}D_2^{be}D_3^{cf}$ is a element in $A$, thus $A$ is closed with respect to functional composition.  We now can say $A$ is a group.
\end{exercise}

\begin{exercise}{5.23}
Show that if $H$ is a subgroup of $S_n$, then either every member of $H$ is an even permutation or exactly half of the members are even.  (This exercise is referred to in Chapter 25.)\\
Suppose $H$ is a subgroup of $S_n$.\\
Suppose that not every member of $H$ is an even permutation and not exactly half of the members are even.\\
Suppose that there are more evens than odds.  Note that there exists at least one odd permutation in $H$, call it $a$.  Note that $a$ applied to any even element produces a odd element of $H$.  Noting that there are more evens than odds we can say there exist two distinct even permutations in $H$ that go to the same odd permutation when $a$ is applied to them (pigeonhole principal), call these elements $b,c$.  Note that $b\neq c$ and yet $ab=ac$, witch via cancellation gives us $b=c$, a contradiction, conclude the negation of our supposition, there are more odds than evens(there can not be the same number by one of our above suppositions).\\
Note that $a$ applied to any odd element produces a even element of $H$.  Noting that there are more odds than evens we can say there exist two distinct odd permutations in $H$ that go to the same even permutation when $a$ is applied to them (pigeonhole principal), call these elements $d,e$.  Note that $d\neq e$ and yet $ad=ae$, witch via cancellation gives us $d=e$, a contradiction, conclude the negation of our supposition, that either every member of $H$ is an even permutation or exactly half of the members are even.
\end{exercise}

\begin{exercise}{5.27}
Use Table 5.1 to compute the following.
\begin{enumerate}[a.]
\item
The centralizer of $\alpha_3 = (13)(24)$\\
$C(\alpha_3)=\{\alpha_1,\alpha_2,\alpha_3,\alpha_4\}$
\item
The centralizer of $a_12 = (124)$\\
$C(\alpha_{12})=\{\alpha_1,\alpha_7,\alpha_{12}\}$
\end{enumerate}
\end{exercise}

\begin{exercise}{5.32}
Let $\beta= (123)(145)$. Write $\beta^{99}$ in disjoint cycle form.\\
Note that $\beta=(14523)$.  Note that $\beta^{99}=(\beta^5)^{19}\beta^4=(e)^{19}(\beta^2)^2=(15342)^2=(13254)$
\end{exercise}

\begin{exercise}{5.38}
Let $H = \{\beta \in S_5 \mid \beta(1) = 1 \text{ and } \beta(3) = 3\}$. Prove that $H$ is a subgroup of $S_5$ . How many elements are in $H$? Is your argument valid when $S_5$ is replaced by $S_n$ for $n \geq 3$? How many elements are in $H$ when $S_5$ is replaced by $A_n$ for $n \geq 4$?\\
To prove $H$ is a subgroup we need only demonstrate closure.  Take $\beta_1,\beta_2\in H\subseteq S_5$.  Note that $\beta_1\beta_2\in S_5$ since $S_5$ has closure.  Note that $\beta_1\beta_2(1)=\beta_1(1)=1$ and $\beta_1\beta_2(3)=\beta_1(3)=3$, thus $\beta_1\beta_2\in H$.  We conclude $H$ is a subgroup.\\
Elements in $H$ are allowed to permute all elements except the first and third, this would be $3$ elements and so there are $3!$ ways to permute them, we conclude $|H|=3!=6$.\\
This argument holds as well if we replace $S_5$ with $S_n$ for $n \geq 3$.  In this case $|H|=(n-2)!$.\\
Noting that the elements in $H$ if we replace $S_5$ with $A_n$ for $n \geq 4$ are simply the even elements of $H$ if we replace $S_5$ with $S_n$, and recalling exercise 5.23 (noting that there is at least one odd element-(24)), we conclude there are exactly half as many elements in $H$ if we replace $S_5$ with $A_n$ as we would have if we replace $S_5$ with $S_n$, that is $\frac{(n-2)!}{2}$.
\end{exercise}





\end{document}