\documentclass[12pt]{article}

\usepackage{amssymb,amsmath,amsthm}
\usepackage[top=1in, bottom=1in, left=1.25in, right=1.25in]{geometry}
\usepackage{fancyhdr}
\usepackage{enumerate}
\usepackage[bw,framed,numbered]{mcode}
\usepackage{graphicx}

% Comment the following line to use TeX's default font of Computer Modern.
\usepackage{times,txfonts}

\newtheoremstyle{homework}% name of the style to be used
  {18pt}% measure of space to leave above the theorem. E.g.: 3pt
  {12pt}% measure of space to leave below the theorem. E.g.: 3pt
  {}% name of font to use in the body of the theorem
  {}% measure of space to indent
  {\bfseries}% name of head font
  {:}% punctuation between head and body
  {2ex}% space after theorem head; " " = normal interword space
  {}% Manually specify head
\theoremstyle{homework} 

% Set up an Exercise environment and a Solution label.
\newtheorem*{exercisecore}{Exercise \@currentlabel}
\newenvironment{exercise}[1]
{\def\@currentlabel{#1}\exercisecore}
{\endexercisecore}

\newcommand{\localhead}[1]{\par\smallskip\noindent\textbf{#1}\nobreak\\}%
\newcommand\solution{\localhead{Solution:}}

%%%%%%%%%%%%%%%%%%%%%%%%%%%%%%%%%%%%%%%%%%%%%%%%%%%%%%%%%%%%%%%%%%%%%%%%
%
% Stuff for getting the name/document date/title across the header
\makeatletter
\RequirePackage{fancyhdr}
\pagestyle{fancy}
\fancyfoot[C]{\ifnum \value{page} > 1\relax\thepage\fi}
\fancyhead[L]{\ifx\@doclabel\@empty\else\@doclabel\fi}
\fancyhead[C]{\ifx\@docdate\@empty\else\@docdate\fi}
\fancyhead[R]{\ifx\@docauthor\@empty\else\@docauthor\fi}
\headheight 15pt

\def\doclabel#1{\gdef\@doclabel{#1}}
\doclabel{Use {\tt\textbackslash doclabel\{MY LABEL\}}.}
\def\docdate#1{\gdef\@docdate{#1}}
\docdate{Use {\tt\textbackslash docdate\{MY DATE\}}.}
\def\docauthor#1{\gdef\@docauthor{#1}}
\docauthor{Use {\tt\textbackslash docauthor\{MY NAME\}}.}
\makeatother

% Shortcuts for blackboard bold number sets (reals, integers, etc.)
\newcommand{\Reals}{\ensuremath{\mathbb R}}
\newcommand{\Nats}{\ensuremath{\mathbb N}}
\newcommand{\Ints}{\ensuremath{\mathbb Z}}
\newcommand{\Rats}{\ensuremath{\mathbb Q}}
\newcommand{\Cplx}{\ensuremath{\mathbb C}}
\newcommand{\Aut}{\ensuremath{\text{Aut}}}
%% Some equivalents that some people may prefer.
\let\RR\Reals
\let\NN\Nats
\let\II\Ints
\let\CC\Cplx

%%%%%%%%%%%%%%%%%%%%%%%%%%%%%%%%%%%%%%%%%%%%%%%%%%%%%%%%%%%%%%%%%%%%%%%%%%%%%%%%%%%%%%%
%%%%%%%%%%%%%%%%%%%%%%%%%%%%%%%%%%%%%%%%%%%%%%%%%%%%%%%%%%%%%%%%%%%%%%%%%%%%%%%%%%%%%%%
% 
% The main document start here.

% The following commands set up the material that appears in the header.

%%%%%%%%%%%%%%%%%%%%%%%%%%%%%%%%%%%%%%%%%%%%%%%%%%%%%%%%%%%%%%%%%%%%%%%%%%%%%%%%%%%%%%%
%%%%%%%%%%%%%%%%%%%%%%%%%%%%%%%%%%%%%%%%%%%%%%%%%%%%%%%%%%%%%%%%%%%%%%%%%%%%%%%%%%%%%%%
% 
% The main document start here.

% The following commands set up the material that appears in the header.
\doclabel{Abstract hw 11-1}
\docauthor{Parker Whaley}
\docdate{March 30 2017}
%\lstinputlisting{} 
\newcommand{\vv}{\mathbf{v}}
\begin{document}
\begin{exercise}{10.16}
Prove that there is no homomorphism from $Z_8\bigoplus  Z_2$ onto $Z_4 \bigoplus Z_4$.\\
Suppose there exists $\phi$ a homomorphism from $Z_8\bigoplus  Z_2$ onto $Z_4 \bigoplus Z_4$.  Noting that both $Z_8\bigoplus  Z_2$ and $Z_4 \bigoplus Z_4$ have 16 elements we can conclude $\phi$ is one-to-one and thus $\phi$ is a isomorphisim.  Note that $(1,0)$ in $Z_8\bigoplus  Z_2$ has order 8.  There must exist some element in $Z_4 \bigoplus Z_4$ that is order 8.  Note that every element of $Z_4 \bigoplus Z_4$ has order 4 or less.  We have reached a contradiction and conclude there is no homomorphism from $Z_8\bigoplus  Z_2$ onto $Z_4 \bigoplus Z_4$.
\end{exercise}

\begin{exercise}{10.20}
How many homomorphisms are there from $Z_{20}$ onto $Z_8$? How many are there to $Z_8$?\\
The kernel of any homomorphisim from $Z_{20}$ onto $Z_8$ must have $20/8$ elements since this is not a integer no such homomorphisim exists.\\
The kernel of a homomorphisim from $Z_{20}$ onto $Z_{4}\leq Z_8$ would have $20/4=5$ elements.  Noting that if such a homomorphisim $\phi$ existed we would require the kernel to be a sub-group and noting that there is only one sub-group of order 5 in $Z_{20}$ we can say that there is only one possible kernel $H=ker(\phi)=\{0,4,8,12,16\}$.  We are now looking for isomorphisims from $Z_{20}/H$ to $Z_{4}$.  There are two, $1+H\rightarrow 1$ and $3+H\rightarrow 1$, thus there are two homomorphisims from $Z_{20}$ onto $Z_{4}$.\\
Under a similar procedure regarding $Z_2$ we find another homomorphisim and there is the homomorphisim $\phi(x)=0$, thus there are a total of 4 homomorphisims from $Z_{20}$ to $Z_8$.
\end{exercise}

\begin{exercise}{10.22}
Suppose that $\phi$ is a homomorphism from a finite group $G$ onto $\bar{G}$ and that $\bar{G}$ has an element of order 8. Prove that $G$ has an element of order 8. Generalize.\\
There exists $\bar{x}\in \bar{G}$ such that $|\bar{x}|=8$.  There exists $x\in G$ such that $\phi(x)=\bar{x}$.  Let $n=|x|$.  Note that $8|n$ thus $8k=n$ for some integer $k$.  Consider $x^k$.  Note that $(x^k)^r$ where $r<8$ cannot be identity since $kr<n$.  Note that $(x^k)^8=x^n=e$ thus $|x^k|=8$.
\end{exercise}

\begin{exercise}{10.24}
Suppose that $\phi: Z_{50} \rightarrow Z_{15}$ is a group homomorphism with $\phi(7) = 6$.\\
\begin{enumerate}[a.]
\item
Determine $\phi(x)$.\\
Note that $\phi(x)=\phi(1^x)=\phi((7^{43})^x)=\phi(7^{43x})=\phi(7)^{43x}=6^{43x}=6*43*x\text{ mod }15=3*x\text{ mod }15$.
\item
Determine the image of $\phi$.\\
Noting that $\phi(x)=3^x$ the image of $\phi$ will be a subset of $<3>$ since it can only generate things of the form $3^x$.  Note that $\phi(\{0,1,2,3,4\})=<3>$ we can say that the image of $\phi$ is precisely $<3>=\{0,3,6,9,12\}$.
\item
Determine the kernel of $\phi$.\\
Note that $x\in \ker(\phi)$ iff $3^x=e_{15}$.  Thus $\ker(\phi)=\{0,5,10,15,20,25,30,35,40,45\}$.
\item
Determine $\phi^{-1}(3)$. That is, determine the set of all elements that map to 3.\\
Note that $\phi(1)=3$ thus $\phi^{-1}(3)=1+\ker(\phi)$.
\end{enumerate}
\end{exercise}

\begin{exercise}{10.26}
Determine all homomorphisms from $Z_4$ to $Z_2 \bigoplus Z_2$.\\
The order of the kernel divides 4 thus there are 3 possible orders for the kernel.\\
The order is 4.  In this case all elements are in the kernel and we have the trivial homomorphism.\\
The order is 2.  Noting that the kernel is a subgroup we can say $\ker(\phi)=\{0,2\}$.  Thus all that is left to decide is where we map $\{1,3\}$ there are three possibilities, mapping to $(1,0)$ or $(0,1)$ or $(1,1)$, and on inspection all three work.\\
The order is 1.  in this case we would have a isomorphisim and noting that one group is cyclic and the other is not this is impossible.\\
We have described all 4 homomorphisms.
\end{exercise}

\begin{exercise}{10.28}
Suppose that $\phi$ is a homomorphism from $S_4$ onto $Z_2$. Determine $\ker \phi$. Determine all homomorphisms from $S_4$ to $Z_2$.\\
The order of $\ker \phi=|S_4|/Z_2=12$.  Noting that there is only one sub group of order $12$ in $S_4$ and that is $A_4$.  Thus we have the homomorphism $\phi(A_4)=0$ and $\phi(S_4-A_4)=1$.  The only other homomorphism would be the trivial homomorphism $\phi(S_4)=0$.
\end{exercise}

\begin{exercise}{10.30}
Suppose that $\phi$ is a homomorphism from a group $G$ onto $Z_6 \bigoplus Z_2$ and that the kernel of $\phi$ has order $5$. Explain why $G$ must have normal subgroups of orders 5, 10, 15, 20, 30, and 60.\\
Recall that if $H\leq Z_6 \bigoplus Z_2$ then $\phi^{-1}(H)\leq G$.  Note that $|\phi^{-1}(H)|=|\ker\phi|*|H|$.\\
\begin{tabular}{|l|l|l|}
\hline
$H$ & $|H|$ & $|\phi^{-1}(H)|$\\
\hline
$(0,0)$ & 1 & 5\\
\hline
$<(0,1)>$ & 2 & 10\\
\hline
$<(1,0)>$ & 6 & 30\\
\hline
$<(2,0)>$ & 3 & 15\\
\hline
$Z_6\bigoplus Z_2$ & 2 & 10\\
\hline
$<3>\bigoplus Z_2$ & 4 & 20\\
\hline
\end{tabular}\\
From the table above we can see the groups in $G$ of the orders described.
\end{exercise}

\end{document}