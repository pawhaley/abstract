\documentclass[12pt]{article}

\usepackage{amssymb,amsmath,amsthm}
\usepackage[top=1in, bottom=1in, left=1.25in, right=1.25in]{geometry}
\usepackage{fancyhdr}
\usepackage{enumerate}
\usepackage[bw,framed,numbered]{mcode}
\usepackage{graphicx}

% Comment the following line to use TeX's default font of Computer Modern.
\usepackage{times,txfonts}

\newtheoremstyle{homework}% name of the style to be used
  {18pt}% measure of space to leave above the theorem. E.g.: 3pt
  {12pt}% measure of space to leave below the theorem. E.g.: 3pt
  {}% name of font to use in the body of the theorem
  {}% measure of space to indent
  {\bfseries}% name of head font
  {:}% punctuation between head and body
  {2ex}% space after theorem head; " " = normal interword space
  {}% Manually specify head
\theoremstyle{homework} 

% Set up an Exercise environment and a Solution label.
\newtheorem*{exercisecore}{Exercise \@currentlabel}
\newenvironment{exercise}[1]
{\def\@currentlabel{#1}\exercisecore}
{\endexercisecore}

\newcommand{\localhead}[1]{\par\smallskip\noindent\textbf{#1}\nobreak\\}%
\newcommand\solution{\localhead{Solution:}}

%%%%%%%%%%%%%%%%%%%%%%%%%%%%%%%%%%%%%%%%%%%%%%%%%%%%%%%%%%%%%%%%%%%%%%%%
%
% Stuff for getting the name/document date/title across the header
\makeatletter
\RequirePackage{fancyhdr}
\pagestyle{fancy}
\fancyfoot[C]{\ifnum \value{page} > 1\relax\thepage\fi}
\fancyhead[L]{\ifx\@doclabel\@empty\else\@doclabel\fi}
\fancyhead[C]{\ifx\@docdate\@empty\else\@docdate\fi}
\fancyhead[R]{\ifx\@docauthor\@empty\else\@docauthor\fi}
\headheight 15pt

\def\doclabel#1{\gdef\@doclabel{#1}}
\doclabel{Use {\tt\textbackslash doclabel\{MY LABEL\}}.}
\def\docdate#1{\gdef\@docdate{#1}}
\docdate{Use {\tt\textbackslash docdate\{MY DATE\}}.}
\def\docauthor#1{\gdef\@docauthor{#1}}
\docauthor{Use {\tt\textbackslash docauthor\{MY NAME\}}.}
\makeatother

% Shortcuts for blackboard bold number sets (reals, integers, etc.)
\newcommand{\Reals}{\ensuremath{\mathbb R}}
\newcommand{\Nats}{\ensuremath{\mathbb N}}
\newcommand{\Ints}{\ensuremath{\mathbb Z}}
\newcommand{\Rats}{\ensuremath{\mathbb Q}}
\newcommand{\Cplx}{\ensuremath{\mathbb C}}
\newcommand{\Aut}{\ensuremath{\text{Aut}}}
%% Some equivalents that some people may prefer.
\let\RR\Reals
\let\NN\Nats
\let\II\Ints
\let\CC\Cplx

%%%%%%%%%%%%%%%%%%%%%%%%%%%%%%%%%%%%%%%%%%%%%%%%%%%%%%%%%%%%%%%%%%%%%%%%%%%%%%%%%%%%%%%
%%%%%%%%%%%%%%%%%%%%%%%%%%%%%%%%%%%%%%%%%%%%%%%%%%%%%%%%%%%%%%%%%%%%%%%%%%%%%%%%%%%%%%%
% 
% The main document start here.

% The following commands set up the material that appears in the header.

%%%%%%%%%%%%%%%%%%%%%%%%%%%%%%%%%%%%%%%%%%%%%%%%%%%%%%%%%%%%%%%%%%%%%%%%%%%%%%%%%%%%%%%
%%%%%%%%%%%%%%%%%%%%%%%%%%%%%%%%%%%%%%%%%%%%%%%%%%%%%%%%%%%%%%%%%%%%%%%%%%%%%%%%%%%%%%%
% 
% The main document start here.

% The following commands set up the material that appears in the header.
\doclabel{Abstract hw 13-2}
\docauthor{Parker Whaley}
\docdate{Apr 14 2017}
%\lstinputlisting{}
\newcommand{\vv}{\mathbf{v}}
\begin{document}

\begin{exercise}{14.6}
Find all maximal ideals in\\
\begin{enumerate}[a.]
\item
$Z_8$\\
$\{0,2,4,6\}$
\item
$Z_{10}$\\
$\{0,5\},\{0,2,4,6,8\}$
\item
$Z_{12}$\\
$\{0,2,4,6,8,10\},\{0,3,6,9\}$
\item
$Z_n$\\
If $n=p_1^{k_1}p_2^{k_2}\cdots$ in terms of prime factorization then $<p_1>_+,<p_2>_+,\cdots$ are all maximal ideals in $Z_n$.
\end{enumerate}
\end{exercise}

\begin{exercise}{14.8}
Prove that the intersection of any set of ideals of a ring is an ideal.\\
Suppose ring $R$ with ideals $A$ and $B$.  Choose some arbitrary element $r\in R$ and choose $x\in A\cap B$.  Note that $rx\in A$ and $rx\in B$ thus $rx\in A\cap B$ and thus $A\cap B$ is an ideal.
\end{exercise}

\begin{exercise}{14.11}
In the ring of integers, find a positive integer $a$ such that\\
\begin{enumerate}[a.]
\item
$<a>=<2>+<3>$\\
$a=1$
\item
$<a>=<6>+<8>$\\
$a=2$
\item
$<a>=<m>+<n>$\\
$a=\gcd(m,n)$
\end{enumerate}
\end{exercise}

\begin{exercise}{14.13}
Find a positive integer $a$ such that\\
\begin{enumerate}[a.]
\item
$<a>=<3><4>$\\
$a=12$
\item
$<a>=<6><8>$\\
$a=48$
\item
$<a>=<m><n>$\\
$a=mn$
\end{enumerate}
\end{exercise}

\begin{exercise}{14.18}
Suppose that in the ring $Z$, the ideal $<35>$ is a proper ideal of $J$ and $J$ is a proper ideal of $I$. What are the possibilities for $J$? What are the possibilities for $I$?\\
$J=<35a>$ where $a$ is some natural bigger than 1 and $I=<35ab>$ where $b$ is some natural bigger than 1.
\end{exercise}

\begin{exercise}{14.22}
Let $I = <2>$. Prove that $I[x]$ is not a maximal ideal of $Z[x]$ even though $I$ is a maximal ideal of $Z$.\\
If we were to add anything to $I$ to try to find a ideal $J$ containing $I$ it must be a odd integer, thus if we simply subtract off the previous integer,witch is even and thus in $I$ and $J$ we conclude that $1\in J$ and thus $J=Z$.  We can now note that $I$ is maximal.\\
Note that $I[x]$ is all polynomials with even coefficients.  Recall that the set of all polynomials where the constant term is even is a ideal of $Z[x]$ and since $I[x]$ is a subset of that set witch is itself not $Z[x]$ we conclude that $I[x]$ is not a maximal ideal of $Z[x]$.
\end{exercise}

\begin{exercise}{14.35}
In $R=Z \bigoplus Z$, let $I = \{(a, 0) \mid a \in Z\}$. Show that $I$ is a prime ideal but
not a maximal ideal.\\
Suppose $(a,b)\in R$ and $(c,0)\in I$.  Note that $(a,b)(c,0)=(ac,0)\in I$ thus $I$ is ideal.\\
Suppose $(a,b),(c,d)\in R$.  Further suppose that $(a,b)(c,d)\in I$.  Note that ether $b=0$ or $d=0$.  WLoG take $b=0$.  Note that $(a,b)=(a,0)\in I$.  Thus $I$ is prime.\\
Note that $J=\{(a,2b)\mid a,b\in Z\}$ is ideal and that $I\subset J\subset R$.  Note that $I$ is not maximal.
\end{exercise}

\end{document}