\documentclass[12pt]{article}

\usepackage{amssymb,amsmath,amsthm}
\usepackage[top=1in, bottom=1in, left=1.25in, right=1.25in]{geometry}
\usepackage{fancyhdr}
\usepackage{enumerate}
\usepackage[bw,framed,numbered]{mcode}
\usepackage{graphicx}

% Comment the following line to use TeX's default font of Computer Modern.
\usepackage{times,txfonts}

\newtheoremstyle{homework}% name of the style to be used
  {18pt}% measure of space to leave above the theorem. E.g.: 3pt
  {12pt}% measure of space to leave below the theorem. E.g.: 3pt
  {}% name of font to use in the body of the theorem
  {}% measure of space to indent
  {\bfseries}% name of head font
  {:}% punctuation between head and body
  {2ex}% space after theorem head; " " = normal interword space
  {}% Manually specify head
\theoremstyle{homework} 

% Set up an Exercise environment and a Solution label.
\newtheorem*{exercisecore}{Exercise \@currentlabel}
\newenvironment{exercise}[1]
{\def\@currentlabel{#1}\exercisecore}
{\endexercisecore}

\newcommand{\localhead}[1]{\par\smallskip\noindent\textbf{#1}\nobreak\\}%
\newcommand\solution{\localhead{Solution:}}

%%%%%%%%%%%%%%%%%%%%%%%%%%%%%%%%%%%%%%%%%%%%%%%%%%%%%%%%%%%%%%%%%%%%%%%%
%
% Stuff for getting the name/document date/title across the header
\makeatletter
\RequirePackage{fancyhdr}
\pagestyle{fancy}
\fancyfoot[C]{\ifnum \value{page} > 1\relax\thepage\fi}
\fancyhead[L]{\ifx\@doclabel\@empty\else\@doclabel\fi}
\fancyhead[C]{\ifx\@docdate\@empty\else\@docdate\fi}
\fancyhead[R]{\ifx\@docauthor\@empty\else\@docauthor\fi}
\headheight 15pt

\def\doclabel#1{\gdef\@doclabel{#1}}
\doclabel{Use {\tt\textbackslash doclabel\{MY LABEL\}}.}
\def\docdate#1{\gdef\@docdate{#1}}
\docdate{Use {\tt\textbackslash docdate\{MY DATE\}}.}
\def\docauthor#1{\gdef\@docauthor{#1}}
\docauthor{Use {\tt\textbackslash docauthor\{MY NAME\}}.}
\makeatother

% Shortcuts for blackboard bold number sets (reals, integers, etc.)
\newcommand{\Reals}{\ensuremath{\mathbb R}}
\newcommand{\Nats}{\ensuremath{\mathbb N}}
\newcommand{\Ints}{\ensuremath{\mathbb Z}}
\newcommand{\Rats}{\ensuremath{\mathbb Q}}
\newcommand{\Cplx}{\ensuremath{\mathbb C}}
\newcommand{\Aut}{\ensuremath{\text{Aut}}}
%% Some equivalents that some people may prefer.
\let\RR\Reals
\let\NN\Nats
\let\II\Ints
\let\CC\Cplx

%%%%%%%%%%%%%%%%%%%%%%%%%%%%%%%%%%%%%%%%%%%%%%%%%%%%%%%%%%%%%%%%%%%%%%%%%%%%%%%%%%%%%%%
%%%%%%%%%%%%%%%%%%%%%%%%%%%%%%%%%%%%%%%%%%%%%%%%%%%%%%%%%%%%%%%%%%%%%%%%%%%%%%%%%%%%%%%
% 
% The main document start here.

% The following commands set up the material that appears in the header.

%%%%%%%%%%%%%%%%%%%%%%%%%%%%%%%%%%%%%%%%%%%%%%%%%%%%%%%%%%%%%%%%%%%%%%%%%%%%%%%%%%%%%%%
%%%%%%%%%%%%%%%%%%%%%%%%%%%%%%%%%%%%%%%%%%%%%%%%%%%%%%%%%%%%%%%%%%%%%%%%%%%%%%%%%%%%%%%
% 
% The main document start here.

% The following commands set up the material that appears in the header.
\doclabel{Abstract hw 13-1}
\docauthor{Parker Whaley}
\docdate{Apr 13 2017}
%\lstinputlisting{}
\newcommand{\vv}{\mathbf{v}}
\begin{document}

\begin{exercise}{13.15}
Let $a$ belong to a ring $R$ with unity and suppose that $a^n = 0$ for some positive integer $n$. (Such an element is called nilpotent.) Prove that $1 - a$ has a multiplicative inverse in $R$. [Hint: Consider $(1 - a)(1 + a + a^2 +\cdots +a^{n-1} )$.]\\
Note that $b=(1 + a + a^2 +\cdots +a^{n-1} )\in R$ and that $(1-a)b=(1 + a + a^2 +\cdots +a^{n-1} )-(a + a^2 + a^3 +\cdots +a^{n} )=1-a^n=1$ thus $b=(1-a)^{-1}$.
\end{exercise}

\begin{exercise}{13.18}
A ring element $a$ is called an idempotent if $a^2 = a$. Prove that the only idempotents in an integral domain are $0$ and $1$.\\
Suppose $R$ is a integral domain.  Suppose that $a\not\in \{0,1\}$ and that $a^2=a$.  Note that $a^{-1}$ exists.  Note that $1=aa^{-1}=a^2a^{-1}=aaa^{-1}=a$ a contradiction we now conclude that the only idempotents in an integral domain are $0$ and $1$.
\end{exercise}

\begin{exercise}{13.22}
Prove that if $a$ is a ring idempotent, then $a^n=a$ for all positive integers $n$.\\
I will proceed with proof by induction.\\
Note that the statement is true for $n=1$ and true for $n=2$.\\
Suppose that the statement $a^n=a$ holds for $n\geq 2$.  Note that $a=a^n=aa^{n-1}=a^2a^{n-1}=a^{n+1}$.  By induction we conclude that if $a$ is a ring idempotent, then $a^n=a$ for all positive integers $n$.
\end{exercise}

\begin{exercise}{13.25}
Find an idempotent in $Z_5 [i] = \{a + bi \mid a, b \in Z_5 \}$.\\
Note that $(3+i)^2=8+6i=3+i$, thus $3+i$ is a idempotent in $Z_5 [i] = \{a + bi \mid a, b \in Z_5 \}$.
\end{exercise}

\begin{exercise}{13.28}
Let $R$ be the set of all real-valued functions defined for all real numbers under function addition and multiplication.
\begin{enumerate}[a)]
\item
Determine all zero-divisors of $R$.\\
The function $f$ is a zero-divisors of $R$ if there exists some $a\in \mathbb{R}$ such that $f(a)=0$.\\
Suppose $f\in R-\{0\}$ and there exists $a\in \mathbb{R}$ such that $f(a)=0$.  Define $$g(x)=\begin{cases} 1 & x=a\\ 0 & x\neq a \end{cases}$$  Note that $g\neq 0$.  Note that $f*g=0$ thus $f$ is a zero-divisors of $R$.\\
Suppose $f$ is a never zero function.  Suppose $f$ is a zero-divisors of $R$.  There exists $g\in R-\{0\}$ such that $f*g=0$.  There exists $a\in \mathbb{R}$ such that $g(a)\neq 0$.  Note that $f*g(a)=f(a)*g(a)\neq 0$ thus we have a contradiction and we conclude that no never zero functions are zero-divisors of $R$.
\item
Determine all nilpotent elements of $R$.\\
Suppose $f\in R-\{0\}$ and $f$ is a nilpotent element of $R$.  There exists some $n$ such that $f^n=0$.  Note that there exists $a\in\mathbb{R}$ such that $f(a)\neq 0$.  Note that $f^n(a)=(f(a))^n\neq 0$ a contradiction conclude that only $0$ is a nilpotent element of $R$.
\item
Show that every nonzero element is a zero-divisor or a unit.\\
Suppose $f$ is a nonzero element.  Suppose $f$ is not a zero-divisor.  Note that $f$ is a never zero function.  Define a function $g(x)=1/f(x)$.  Note that $f*g=1$.  Conclude $f$ is a unit.  Conclude that every nonzero element is a zero-divisor or a unit.
\end{enumerate}
\end{exercise}

\begin{exercise}{13.31}
Let $R$ be a ring with unity $1$. If the product of any pair of nonzero elements of $R$ is nonzero, prove that $ab = 1$ implies $ba = 1$.\\
Suppose $ab = 1$.  Note that $(ba)b=b(ab)=b$ and thus $ba=1$.
\end{exercise}

\begin{exercise}{13.35}
Let $F$ be a field of order $2^n$. Prove that $\text{char } F = 2$.\\
Suppose that $(\not\exists a\neq 0) a=-a$.  In this case we can pair off non-zero elements each with there additive inverse.  This means that there are a even number of non-zero elements, or that the total number of elements is odd, a contradiction with the total number of elements being $2^n$.\\
We know there must exist some $a\in F-\{0\}$ such that $a=-a$.  Note that $a(1+1)=a+a=a+(-a)=0$ implies that $1+1=0$ and thus $\text{char } F = 2$.
\end{exercise}

\begin{exercise}{13.51}
Show that any finite field has order $p^n$ , where $p$ is a prime. Hint: Use facts about finite Abelian groups. (This exercise is referred to in Chapter 22.)\\
Suppose $F$ is a finite field.  Suppose $|F|$ is not a prime to a power.  Note that $6\leq |F|=abc$ where $a,b$ are distinct primes.  There must exist a element in the finite abelian group $F$ under $+$ such that $|A|_+=a$, and another element such that $|B|_+=b$.  Note that $aA=A+A+\cdots+A=0$ implies $A(1+1+\cdots+1)=0$ and thus since $A$ is clearly non zero $a1=0$ or in other words $|1|_+\mid a$ and likewise $|1|_+\mid b$.  Note that $|1|_+\mid \gcd(a,b)=1$ thus $|1|_+=1$ thus $1=0$ a contradiction, conclude that $|F|=p^n$ for some prime $p$ and some natural $n$.
\end{exercise}

\end{document}