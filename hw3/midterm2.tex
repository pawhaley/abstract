%%%%%%%%%%%%%%%%%%%%%%%%%%%%%%%%%%%%%%%%%%%%%%%%%%%%%%%%%%%%%%%%%%%%%%%%%%%%%%%%%%%%%%%
%%%%%%%%%%%%%%%%%%%%%%%%%%%%%%%%%%%%%%%%%%%%%%%%%%%%%%%%%%%%%%%%%%%%%%%%%%%%%%%%%%%%%%%
% 
% This top part of the document is called the 'preamble'.  Modify it with caution!
%
% The real document starts below where it says 'The main document starts here'.

\documentclass[12pt]{article}

\usepackage{amssymb,amsmath,amsthm}
\usepackage[top=1in, bottom=1in, left=1.25in, right=1.25in]{geometry}
\usepackage{fancyhdr}
\usepackage{enumerate}
\usepackage{color}

% Comment the following line to use TeX's default font of Computer Modern.
\usepackage{times,txfonts}

\newtheoremstyle{homework}% name of the style to be used
  {18pt}% measure of space to leave above the theorem. E.g.: 3pt
  {12pt}% measure of space to leave below the theorem. E.g.: 3pt
  {}% name of font to use in the body of the theorem
  {}% measure of space to indent
  {\bfseries}% name of head font
  {:}% punctuation between head and body
  {2ex}% space after theorem head; " " = normal interword space
  {}% Manually specify head
\theoremstyle{homework} 

% Set up an Exercise environment and a Solution label.
\newtheorem*{exercisecore}{Exercise \@currentlabel}
\newenvironment{exercise}[1]
{\def\@currentlabel{#1}\exercisecore}
{\endexercisecore}

\newcommand\W{{\color{red}\textbf{(W) (Hand this one in to David.)}}}
\newcommand\tome{{\color{red}\textbf{(Hand this one in to David.)}}}

\newcommand{\localhead}[1]{\par\smallskip\noindent\textbf{#1}\nobreak\\}%
\newcommand\solution{\localhead{Solution:}}

%%%%%%%%%%%%%%%%%%%%%%%%%%%%%%%%%%%%%%%%%%%%%%%%%%%%%%%%%%%%%%%%%%%%%%%%
%
% Stuff for getting the name/document date/title across the header
\makeatletter
\RequirePackage{fancyhdr}
\pagestyle{fancy}
\fancyfoot[C]{\ifnum \value{page} > 1\relax\thepage\fi}
\fancyhead[L]{\ifx\@doclabel\@empty\else\@doclabel\fi}
\fancyhead[C]{\ifx\@docdate\@empty\else\@docdate\fi}
\fancyhead[R]{\ifx\@docauthor\@empty\else\@docauthor\fi}
\headheight 15pt

\def\doclabel#1{\gdef\@doclabel{#1}}
\doclabel{Use {\tt\textbackslash doclabel\{MY LABEL\}}.}
\def\docdate#1{\gdef\@docdate{#1}}
\docdate{Use {\tt\textbackslash docdate\{MY DATE\}}.}
\def\docauthor#1{\gdef\@docauthor{#1}}
\docauthor{Use {\tt\textbackslash docauthor\{MY NAME\}}.}
\makeatother

% Shortcuts for blackboard bold number sets (reals, integers, etc.)
\newcommand{\Reals}{\ensuremath{\mathbb R}}
\newcommand{\Nats}{\ensuremath{\mathbb N}}
\newcommand{\Ints}{\ensuremath{\mathbb Z}}
\newcommand{\Rats}{\ensuremath{\mathbb Q}}
\newcommand{\Cplx}{\ensuremath{\mathbb C}}
%% Some equivalents that some people may prefer.
\let\RR\Reals
\let\NN\Nats
\let\II\Ints
\let\CC\Cplx

%%%%%%%%%%%%%%%%%%%%%%%%%%%%%%%%%%%%%%%%%%%%%%%%%%%%%%%%%%%%%%%%%%%%%%%%%%%%%%%%%%%%%%%
%%%%%%%%%%%%%%%%%%%%%%%%%%%%%%%%%%%%%%%%%%%%%%%%%%%%%%%%%%%%%%%%%%%%%%%%%%%%%%%%%%%%%%%
% 
% The main document start here.

% The following commands set up the material that appears in the header.
\doclabel{Math 405: HW 3}
\docauthor{Parker Whaley}
\docdate{Due Jan 27, 2017}

\begin{document}
\begin{exercise}
{2.11}
Find the inverse of the element $\begin{bmatrix}
2&6\\ 3&5
\end{bmatrix} $ in $GL(2, Z_{11} )$.\\
The inverse to this matrix would be $\frac{1}{3}\begin{bmatrix}
5&-6\\ -3&2
\end{bmatrix}\mod 11= 4\begin{bmatrix}
5&5\\ 8&2
\end{bmatrix}\mod 11=
\begin{bmatrix}
9&9\\
10&8
\end{bmatrix}$
\end{exercise}

\begin{exercise}
{2.13}
Translate each of the following multiplicative expressions into its additive counterpart. Assume that the operation is commutative.
\begin{enumerate}[(a)]
\item
$a^2 b^3$ translates to $2a+3b$.
\item
$a^{-2} (b^{-1} c)^2$ translates to $-2a+2(-b+c)$.
\item
$(ab^2 )^{-3} c^2 = e$ translates to $-3(a+2b ) +2c = e$.
\end{enumerate}
\end{exercise}

\begin{exercise}
{2.19}
Prove that the set of all $2 \times 2$ matrices with entries from $\mathbb{R}$ and determinant $+1$ is a group under matrix multiplication.\\
I will now go through the definition of a group:
\begin{enumerate}[(1)]
\item
Associativity\\
We have previously discussed that all matrices have this property under multiplication.
\item
Identity\\
Note that $\begin{bmatrix}
1&0\\
0&1
\end{bmatrix}$ acts as Identity on this set.
\item
Inverse\\
Note that given a arbitrary matrix with determinate of $1$, $\begin{bmatrix}
a&b\\ c&d
\end{bmatrix}$, the matrix $\begin{bmatrix}
a&b\\ c&d
\end{bmatrix}\cdot \begin{bmatrix}
d&-b\\ -c&a
\end{bmatrix}=\begin{bmatrix}
ad-bc&0\\ 0&ad-bc
\end{bmatrix}=\begin{bmatrix}
1&0\\ 0&1
\end{bmatrix}$
\item
Closure\\
Suppose that the matrices $\begin{bmatrix}
a&b\\ c&d
\end{bmatrix}$ and $\begin{bmatrix}
e&f\\ g&h
\end{bmatrix}$ both are determinate one.  Note that there multiple $\begin{bmatrix}
a&b\\ c&d
\end{bmatrix}\cdot \begin{bmatrix}
e&f\\ g&h
\end{bmatrix}=\begin{bmatrix}
ae+bg & af+bh\\ ce+dg & cf+dh
\end{bmatrix}$, has a determinant of $(ae+bg)(cf+dh)-(af+bh)(ce+dg)=aecf+bgcf+aedh+bgdh-afce-bhce-afdg-bhdg=bgcf+aedh-bhce-afdg=bc(gf-he)+ad(eh-fg)=bc(-1)+ad(1)=1$, thus closure.

\end{enumerate}
\end{exercise}

\begin{exercise}
{2.22}
Let $G$ be a group with the property that for any $x$, $y$, $z$ in the group, $xy = zx$ implies $y = z$. Prove that $G$ is Abelian. (“Left-right cancellation” implies commutativity.)\\
Note that $b\cdot ab=ba\cdot b$.  By left-right cancellation $ab=ba$.
\end{exercise}

\begin{exercise}
{2.25}
Prove that a group $G$ is Abelian if and only if $(ab)^{-1}=a^{-1}b^{-1}$ for all $a$ and $b$ in $G$.\\
Suppose $G$ is Abelian.  Consider $a$ and $b$, elements of $G$.  Note that $a^{-1}$ and $b^{-1}$ are in $G$.  Note that $(ab)^{-1}=b^{-1}a^{-1}=a^{-1}b^{-1}$.\\
Suppose $(ab)^{-1}=a^{-1}b^{-1}$, for all $a$ and $b$ in $G$.  Consider $a$ and $b$, elements of $G$.  Note that $a^{-1}$ and $b^{-1}$ are in $G$.  Note that $ab=(a^{-1})^{-1}(b^{-1})^{-1}=(b^{-1}a^{-1})^{-1}=(b^{-1})^{-1}(a^{-1})^{-1}=ba$, thus $G$ is Abelian.
\end{exercise}


\end{document}