\documentclass[12pt]{article}

\usepackage{amssymb,amsmath,amsthm}
\usepackage[top=1in, bottom=1in, left=1.25in, right=1.25in]{geometry}
\usepackage{fancyhdr}
\usepackage{enumerate}
\usepackage[bw,framed,numbered]{mcode}
\usepackage{graphicx}

% Comment the following line to use TeX's default font of Computer Modern.
\usepackage{times,txfonts}

\newtheoremstyle{homework}% name of the style to be used
  {18pt}% measure of space to leave above the theorem. E.g.: 3pt
  {12pt}% measure of space to leave below the theorem. E.g.: 3pt
  {}% name of font to use in the body of the theorem
  {}% measure of space to indent
  {\bfseries}% name of head font
  {:}% punctuation between head and body
  {2ex}% space after theorem head; " " = normal interword space
  {}% Manually specify head
\theoremstyle{homework} 

% Set up an Exercise environment and a Solution label.
\newtheorem*{exercisecore}{Exercise \@currentlabel}
\newenvironment{exercise}[1]
{\def\@currentlabel{#1}\exercisecore}
{\endexercisecore}

\newcommand{\localhead}[1]{\par\smallskip\noindent\textbf{#1}\nobreak\\}%
\newcommand\solution{\localhead{Solution:}}

%%%%%%%%%%%%%%%%%%%%%%%%%%%%%%%%%%%%%%%%%%%%%%%%%%%%%%%%%%%%%%%%%%%%%%%%
%
% Stuff for getting the name/document date/title across the header
\makeatletter
\RequirePackage{fancyhdr}
\pagestyle{fancy}
\fancyfoot[C]{\ifnum \value{page} > 1\relax\thepage\fi}
\fancyhead[L]{\ifx\@doclabel\@empty\else\@doclabel\fi}
\fancyhead[C]{\ifx\@docdate\@empty\else\@docdate\fi}
\fancyhead[R]{\ifx\@docauthor\@empty\else\@docauthor\fi}
\headheight 15pt

\def\doclabel#1{\gdef\@doclabel{#1}}
\doclabel{Use {\tt\textbackslash doclabel\{MY LABEL\}}.}
\def\docdate#1{\gdef\@docdate{#1}}
\docdate{Use {\tt\textbackslash docdate\{MY DATE\}}.}
\def\docauthor#1{\gdef\@docauthor{#1}}
\docauthor{Use {\tt\textbackslash docauthor\{MY NAME\}}.}
\makeatother

% Shortcuts for blackboard bold number sets (reals, integers, etc.)
\newcommand{\Reals}{\ensuremath{\mathbb R}}
\newcommand{\Nats}{\ensuremath{\mathbb N}}
\newcommand{\Ints}{\ensuremath{\mathbb Z}}
\newcommand{\Rats}{\ensuremath{\mathbb Q}}
\newcommand{\Cplx}{\ensuremath{\mathbb C}}
\newcommand{\Aut}{\ensuremath{\text{Aut}}}
%% Some equivalents that some people may prefer.
\let\RR\Reals
\let\NN\Nats
\let\II\Ints
\let\CC\Cplx

%%%%%%%%%%%%%%%%%%%%%%%%%%%%%%%%%%%%%%%%%%%%%%%%%%%%%%%%%%%%%%%%%%%%%%%%%%%%%%%%%%%%%%%
%%%%%%%%%%%%%%%%%%%%%%%%%%%%%%%%%%%%%%%%%%%%%%%%%%%%%%%%%%%%%%%%%%%%%%%%%%%%%%%%%%%%%%%
% 
% The main document start here.

% The following commands set up the material that appears in the header.

%%%%%%%%%%%%%%%%%%%%%%%%%%%%%%%%%%%%%%%%%%%%%%%%%%%%%%%%%%%%%%%%%%%%%%%%%%%%%%%%%%%%%%%
%%%%%%%%%%%%%%%%%%%%%%%%%%%%%%%%%%%%%%%%%%%%%%%%%%%%%%%%%%%%%%%%%%%%%%%%%%%%%%%%%%%%%%%
% 
% The main document start here.

% The following commands set up the material that appears in the header.
\doclabel{Abstract hw 14-2}
\docauthor{Parker Whaley}
\docdate{Apr 28 2017}
%\lstinputlisting{}
\newcommand{\vv}{\mathbf{v}}
\begin{document}

\begin{exercise}{15.42}
Determine all ring homomorphisms from $\mathbb{Q}$ to $\mathbb{Q}$.\\
Suppose $\phi:\mathbb{Q}\rightarrow\mathbb{Q}$ is a homomorphism.  Suppose $\phi(1)=a$.  Note that $a^2=a$ and thus eather $a=1$ or $a=0$.\\
In the case that $a=0$, $\phi(n/m)=\phi(n)*\phi(1/m)=n\phi(1)*\phi(1/m)=n*0*\phi(1/m)=0$.\\
In the case where $a=1$, note that for any integer m, $\phi(m^{-1})*\phi(m)=\phi(1)=1$ and thus $\phi(m^{-1})=\phi(m)^{-1}$.  Note that $\phi(n/m)=\phi(n)[\phi(m)]^{-1}=n\phi(1)[m\phi(1)]^{-1}=n1[m1]^{-1}=n[m]^{-1}=n/m$.\\
We have now found the only two possible homomorphisms, note that both are actually homomorphisms:
$$x\rightarrow 0$$
$$x\rightarrow x$$
\end{exercise}

\begin{exercise}{15.44}
Let $R$ be a commutative ring of prime characteristic $p$. Show that the Frobenius map $\phi : x \rightarrow x^p$ is a ring homomorphism from $R$ to $R$.\\
Note that $\phi(a+b)=(a+b)^p=\sum_{k=0}^p \alpha_k a^kb^{p-k}$ for some integers $\alpha_k$.  Note that the number of copys of $a^kb^{p-k}$ appering here is exactly how many ways we can choose witch term to take the $a$'s from in $(a+b)*\cdots *(a+b)$ witch is ${p}\choose{k}$ (for more info read about Pascal's triangle), thus $\alpha_k={{p}\choose{k}}=\frac{p!}{(p-k)!k!}$.\\
Consider $\alpha_k$ when $k\neq 0$ and $k\neq p$.  In this case note that $p$ will be one of the multiples in $p!$ but not in $(p-k)!k!$ in other words $p$ apears in the prime factorization of the top but not in the prime factorization of the bottom of $\frac{p!}{(p-k)!k!}$ (here we are using the fact that $p$ is prime).  Note that in the reduced form of $\frac{p!}{(p-k)!k!}$, $p$ must still appear in the top thus noting that in the reduced form the botom must be 1 since $\alpha_k$ is a integer we conclude that $p\mid \alpha_k$.  Define $\beta_k$ for $2\leq k\leq p-1$ such that $p\beta_k=\alpha_k$.\\
Now note that $\phi(a+b)=\sum_{k=0}^p \alpha_k a^kb^{p-k}=a^p+b^p+\sum_{k=1}^{p-1} \alpha_k a^kb^{p-k}=a^p+b^p+\sum_{k=1}^{p-1} \beta_k p (a^kb^{p-k})=a^p+b^p+\sum_{k=1}^{p-1} \beta_k 0_R=a^p+b^p=\phi(a)+\phi(b)$, thus $\phi$ preserves addition.\\
Note that $\phi(ab)=[ab]^p=a^pb^p=\phi(a)\phi(b)$ thus $\phi$ is operation preserving under multiplication.  Note that $\phi$ is a homomorphism.
\end{exercise}

\begin{exercise}{15.46}
Show that a homomorphism from a field onto a ring with more than one element must be an isomorphism.\\
Let $F$ be a field and $R\neq \{0_R\}$ be a ring.  Suppose $\phi$ is a homomorphism from $F$ onto $R$.  Note that $\ker(\phi)$ must be a ideal in $F$.  Note that the only ideals in a field are $\{0_F\}$ and $F$.  Choose $a\in R-\{0_R\}$.  Note that there must exist $a'\in F$ such that $\phi(a')=a$ thus $a'\not\in\ker(\phi)$ and thus $\ker(\phi)\neq F$.  Note that $\ker(\phi)=\{0_F\}$, thus $\phi$ is a isomorphism.
\end{exercise}

\begin{exercise}{15.48}
A principal ideal ring is a ring with the property that every ideal has the form $<a>$.  Show that the homomorphic image of a principal ideal ring is a principal ideal ring.\\
Suppose $R$ is a principal ideal ring and $\phi$ is a homomorphisim to some other ring, let $\phi(R)=H$, thus $\phi$ becomes a homomorphisim from $R$ onto $H$.  Let $K=\phi(<a>)=\phi(\{r*a\mid r\in R\})=\{\phi(r*a)\mid r\in R\}=\{\phi(r)*\phi(a)\mid r\in R\}=\{h*\phi(a)\mid h\in \phi(R)\}=\{h*\phi(a)\mid h\in H\}=<\phi(a)>$ in the ring $H$.  Thus the homomorphic image of a principal ideal ring is a principal ideal ring.
\end{exercise}

\begin{exercise}{15.53}
Determine all ring homomorphisms from $\mathbb{R}$ to $\mathbb{R}$.\\
Note that $\mathbb{R}$ is a field and thus the ideals are $\{0\}$ and $\mathbb{R}$.  Suppose $\phi$ is a homomorphisms from $\mathbb{R}$ to $\mathbb{R}$.  Note that $\ker(\phi)$ is ideal and thus $\ker(\phi)=\{0\}$ or $\ker(\phi)=\mathbb{R}$.\\
In the case that $\ker(\phi)=\mathbb{R}$, we only have one possibility for $\phi$ and that is $\phi(x)=0$.\\\\
In the case that $\ker(\phi)=\{0\}$ note that $\phi(1)=1$ and $\phi$ is a isomorphisim.  Note that if $n\in\mathbb{N}\subseteq \mathbb{R}$, $\phi(n)=\phi(n*1)=n\phi(1)=n$ thus $\phi$ acts as identity on $\mathbb{N}$.  Note that if $(-n)\in(-1)*\mathbb{N}\subseteq \mathbb{R}$ then $\phi(-n)+n=\phi(-n)+\phi(n)=\phi(0)=0$ thus $\phi(-n)=-n$ and we know that $\phi$ acts as identity on all of $\mathbb{Z}$.  If $q/p\in\mathbb{Q}\subseteq\mathbb{R}$ then $\phi(q/p)=\phi(q)\phi(p^{-1})=\phi(q)\phi(p)^{-1}=qp^{-1}=q/p$ thus $\phi$ acts as identity on all of $\mathbb{Q}$.
\end{exercise}

\begin{exercise}{15.56}

\end{exercise}

\end{document}