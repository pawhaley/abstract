%%%%%%%%%%%%%%%%%%%%%%%%%%%%%%%%%%%%%%%%%%%%%%%%%%%%%%%%%%%%%%%%%%%%%%%%%%%%%%%%%%%%%%%
%%%%%%%%%%%%%%%%%%%%%%%%%%%%%%%%%%%%%%%%%%%%%%%%%%%%%%%%%%%%%%%%%%%%%%%%%%%%%%%%%%%%%%%
% 
% This top part of the document is called the 'preamble'.  Modify it with caution!
%
% The real document starts below where it says 'The main document starts here'.

\documentclass[12pt]{article}

\usepackage{amssymb,amsmath,amsthm}
\usepackage[top=1in, bottom=1in, left=1.25in, right=1.25in]{geometry}
\usepackage{fancyhdr}
\usepackage{enumerate}
\usepackage{color}

% Comment the following line to use TeX's default font of Computer Modern.
\usepackage{times,txfonts}

\newtheoremstyle{homework}% name of the style to be used
  {18pt}% measure of space to leave above the theorem. E.g.: 3pt
  {12pt}% measure of space to leave below the theorem. E.g.: 3pt
  {}% name of font to use in the body of the theorem
  {}% measure of space to indent
  {\bfseries}% name of head font
  {:}% punctuation between head and body
  {2ex}% space after theorem head; " " = normal interword space
  {}% Manually specify head
\theoremstyle{homework} 

% Set up an Exercise environment and a Solution label.
\newtheorem*{exercisecore}{Exercise \@currentlabel}
\newenvironment{exercise}[1]
{\def\@currentlabel{#1}\exercisecore}
{\endexercisecore}

\newcommand\W{{\color{red}\textbf{(W) (Hand this one in to David.)}}}
\newcommand\tome{{\color{red}\textbf{(Hand this one in to David.)}}}

\newcommand{\localhead}[1]{\par\smallskip\noindent\textbf{#1}\nobreak\\}%
\newcommand\solution{\localhead{Solution:}}

%%%%%%%%%%%%%%%%%%%%%%%%%%%%%%%%%%%%%%%%%%%%%%%%%%%%%%%%%%%%%%%%%%%%%%%%
%
% Stuff for getting the name/document date/title across the header
\makeatletter
\RequirePackage{fancyhdr}
\pagestyle{fancy}
\fancyfoot[C]{\ifnum \value{page} > 1\relax\thepage\fi}
\fancyhead[L]{\ifx\@doclabel\@empty\else\@doclabel\fi}
\fancyhead[C]{\ifx\@docdate\@empty\else\@docdate\fi}
\fancyhead[R]{\ifx\@docauthor\@empty\else\@docauthor\fi}
\headheight 15pt

\def\doclabel#1{\gdef\@doclabel{#1}}
\doclabel{Use {\tt\textbackslash doclabel\{MY LABEL\}}.}
\def\docdate#1{\gdef\@docdate{#1}}
\docdate{Use {\tt\textbackslash docdate\{MY DATE\}}.}
\def\docauthor#1{\gdef\@docauthor{#1}}
\docauthor{Use {\tt\textbackslash docauthor\{MY NAME\}}.}
\makeatother

% Shortcuts for blackboard bold number sets (reals, integers, etc.)
\newcommand{\Reals}{\ensuremath{\mathbb R}}
\newcommand{\Nats}{\ensuremath{\mathbb N}}
\newcommand{\Ints}{\ensuremath{\mathbb Z}}
\newcommand{\Rats}{\ensuremath{\mathbb Q}}
\newcommand{\Cplx}{\ensuremath{\mathbb C}}
%% Some equivalents that some people may prefer.
\let\RR\Reals
\let\NN\Nats
\let\II\Ints
\let\CC\Cplx

%%%%%%%%%%%%%%%%%%%%%%%%%%%%%%%%%%%%%%%%%%%%%%%%%%%%%%%%%%%%%%%%%%%%%%%%%%%%%%%%%%%%%%%
%%%%%%%%%%%%%%%%%%%%%%%%%%%%%%%%%%%%%%%%%%%%%%%%%%%%%%%%%%%%%%%%%%%%%%%%%%%%%%%%%%%%%%%
% 
% The main document start here.

% The following commands set up the material that appears in the header.
\doclabel{Math 405: HW 1}
\docauthor{Parker Whaley}
\docdate{Due Jan 20, 2017}

\begin{document}
\begin{exercise}
6
Suppose a and b are integers that divide the integer c. If a and b are relatively prime, show that ab divides c. Show, by example, that if a and b are not relatively prime, then ab need not divide c.\\
By the fundamental theorem of algebra we can create a prime factorization for a,b, and c.  Note that $a | c$ and $b | c$ implies that c contains the primes of a and b.  Since a and b are relatively prime we know that they share no common prime.  Thus there multiple contains all of the primes of a and those of b to the power of the original in eater a or b.  Thus the primes in c are in ab and so $ab | c$.\\
Consider $a=4$ and $b=6$ and $c=12$.  Note that $a$ and $b$ divide $c$ however $ab=24>c$ and thus $ab$ does not divide $c$
\end{exercise}

\begin{exercise}
7
If $a$ and $b$ are integers and $n$ is a positive integer, prove that $a \mod n =b \mod n$ if and only if $n$ divides $a - b$.\\
Suppose $a \mod n =b \mod n$.  Note that there exists integer $i$, such that, $a=in+(a \mod n)$, also there exists integer $j$ such that $b=jn+(b \mod n)$.  Note that $a-b=in+(a \mod n)-jn-(b \mod n)=in-jn=(i-j)n$ thus $n$ divides $a - b$.\\
Suppose $n$ divides $a - b$.  Note that there exists integer $i$, such that, $a=in+(a \mod n)$, also there exists integer $j$ such that $b=jn+(b \mod n)$.  Note that $a-b=in+(a \mod n)-jn-(b \mod n)=in-jn+(a \mod n)-(b \mod n)=(i-j)n+(a \mod n)-(b \mod n)$.  Note that $(i-j)n+(a \mod n)-(b \mod n)=kn$ for some $k$.  Thus we know that $(a \mod n)-(b \mod n)=vn$ for some integer $v$.  However we also know $-n<(a \mod n)-(b \mod n)<n$, thus $(a \mod n)-(b \mod n)=0$ or $(a \mod n)=(b \mod n)$.
\end{exercise}

\begin{exercise}
9
Let $n$ be a fixed positive integer greater than $1$. If $a \mod n = a'$ and $b \mod n = b'$, prove that $(a - b) \mod n = (a' - b') \mod n$ and $(ab) \mod n = (a'b') \mod n$. (This exercise is referred to in Chapters 6, 8, 10, and 15.)\\
Note there exists $j$ and $k$ such that $a=jn+a'$ and $b=kn+b'$.  Note $(a - b) \mod n = (jn+a' - kn-b') \mod n = (a' - b') \mod n$.  Note that $(ab) \mod n = (jn+a')(kn+b') \mod n = (jnkn+kna'+jnb'+a'b') \mod n = (a'b') \mod n$.
\end{exercise}

\begin{exercise}
{11}
Let $n$ and $a$ be positive integers and let $d = \gcd(a, n)$. Show that the equation $ax \mod n = 1$ has a solution if and only if $d = 1$. (This exercise is referred to in Chapter 2.)\\
Suppose there exists a $x$ such that $ax \mod n = 1$.  Note that there exists some $j$ such that $ax=jn+ax \mod n$.  Note that $ax-jn=1$ thus $\gcd(a, n)=1=d$.
\end{exercise}

\begin{exercise}
{12}
Show that $5n + 3$ and $7n + 4$ are relatively prime for all $n$.\\
Note that $7(5n + 3)-5(7n + 4)=7\cdot 5n + 21-5\cdot 7n -20=1$ thus $5n + 3$ and $7n + 4$ are relatively prime.
\end{exercise}

\begin{exercise}
{13}
Suppose that $m$ and $n$ are relatively prime and $r$ is any integer. Show that there are integers $x$ and $y$ such that $mx + ny = r$.\\
Since $m$ and $n$ are relatively prime there exists integers $j$ and $k$ such that $jm+kn=1$.  Let $x=rj$ and $y=rk$.  Note that $mx + ny=mrj + nrk=r(jm+kn)=r$.
\end{exercise}



\end{document}