\documentclass[12pt]{article}

\usepackage{amssymb,amsmath,amsthm}
\usepackage[top=1in, bottom=1in, left=1.25in, right=1.25in]{geometry}
\usepackage{fancyhdr}
\usepackage{enumerate}
\usepackage[bw,framed,numbered]{mcode}
\usepackage{graphicx}

% Comment the following line to use TeX's default font of Computer Modern.
\usepackage{times,txfonts}

\newtheoremstyle{homework}% name of the style to be used
  {18pt}% measure of space to leave above the theorem. E.g.: 3pt
  {12pt}% measure of space to leave below the theorem. E.g.: 3pt
  {}% name of font to use in the body of the theorem
  {}% measure of space to indent
  {\bfseries}% name of head font
  {:}% punctuation between head and body
  {2ex}% space after theorem head; " " = normal interword space
  {}% Manually specify head
\theoremstyle{homework} 

% Set up an Exercise environment and a Solution label.
\newtheorem*{exercisecore}{Exercise \@currentlabel}
\newenvironment{exercise}[1]
{\def\@currentlabel{#1}\exercisecore}
{\endexercisecore}

\newcommand{\localhead}[1]{\par\smallskip\noindent\textbf{#1}\nobreak\\}%
\newcommand\solution{\localhead{Solution:}}

%%%%%%%%%%%%%%%%%%%%%%%%%%%%%%%%%%%%%%%%%%%%%%%%%%%%%%%%%%%%%%%%%%%%%%%%
%
% Stuff for getting the name/document date/title across the header
\makeatletter
\RequirePackage{fancyhdr}
\pagestyle{fancy}
\fancyfoot[C]{\ifnum \value{page} > 1\relax\thepage\fi}
\fancyhead[L]{\ifx\@doclabel\@empty\else\@doclabel\fi}
\fancyhead[C]{\ifx\@docdate\@empty\else\@docdate\fi}
\fancyhead[R]{\ifx\@docauthor\@empty\else\@docauthor\fi}
\headheight 15pt

\def\doclabel#1{\gdef\@doclabel{#1}}
\doclabel{Use {\tt\textbackslash doclabel\{MY LABEL\}}.}
\def\docdate#1{\gdef\@docdate{#1}}
\docdate{Use {\tt\textbackslash docdate\{MY DATE\}}.}
\def\docauthor#1{\gdef\@docauthor{#1}}
\docauthor{Use {\tt\textbackslash docauthor\{MY NAME\}}.}
\makeatother

% Shortcuts for blackboard bold number sets (reals, integers, etc.)
\newcommand{\Reals}{\ensuremath{\mathbb R}}
\newcommand{\Nats}{\ensuremath{\mathbb N}}
\newcommand{\Ints}{\ensuremath{\mathbb Z}}
\newcommand{\Rats}{\ensuremath{\mathbb Q}}
\newcommand{\Cplx}{\ensuremath{\mathbb C}}
\newcommand{\Aut}{\ensuremath{\text{Aut}}}
%% Some equivalents that some people may prefer.
\let\RR\Reals
\let\NN\Nats
\let\II\Ints
\let\CC\Cplx

%%%%%%%%%%%%%%%%%%%%%%%%%%%%%%%%%%%%%%%%%%%%%%%%%%%%%%%%%%%%%%%%%%%%%%%%%%%%%%%%%%%%%%%
%%%%%%%%%%%%%%%%%%%%%%%%%%%%%%%%%%%%%%%%%%%%%%%%%%%%%%%%%%%%%%%%%%%%%%%%%%%%%%%%%%%%%%%
% 
% The main document start here.

% The following commands set up the material that appears in the header.

%%%%%%%%%%%%%%%%%%%%%%%%%%%%%%%%%%%%%%%%%%%%%%%%%%%%%%%%%%%%%%%%%%%%%%%%%%%%%%%%%%%%%%%
%%%%%%%%%%%%%%%%%%%%%%%%%%%%%%%%%%%%%%%%%%%%%%%%%%%%%%%%%%%%%%%%%%%%%%%%%%%%%%%%%%%%%%%
% 
% The main document start here.

% The following commands set up the material that appears in the header.
\doclabel{Abstract hw 14-1}
\docauthor{Parker Whaley}
\docdate{Apr 27 2017}
%\lstinputlisting{}
\newcommand{\vv}{\mathbf{v}}
\begin{document}
\begin{exercise}{15.12}
Let $Z_3[i] = \{a + bi \mid a, b \in Z_3 \}$ (see Example 9 in Chapter 13). Show that the field $Z_3[i]$ is ring isomorphic to the field $Z_3[x]/<x^2 + 1>$.\\
Define $H=<x^2+1>$.  Define $\phi:Z_3[i]\rightarrow Z_3[x]/<x^2 + 1>$ as $\phi(a+bi)= a+bx+H$.  Note that any element in $Z_3[x]/<x^2 + 1>$ can be written in the form $g(x)=f(x)*(x^2+1)+r(x)$ where $r(x)=a+bx$ is of order at most $x$, simply by noting that $g(x)\in a+bx+H=\phi(a+bi)$ we can say that $\phi$ is onto.  Note that $\phi((a+bi)(c+di))=\phi(ac-bd+(cb+da)i)=ac-bd+(cb+da)x+H$.  Note that $\phi(a+bi)\phi(c+di)=(a+bx+H)(c+dx+H)=ac+(cb+da)x+bdx^2+H=ac+(cb+da)x+bdx^2-bd(x^2+1)+H=ac-bd+(cb+da)x+H$, thus $\phi$ preserves multiplication.  Note that $\phi((a+bi)+(c+di))=\phi(a+bi+c+di)=a+c+(b+d)x+H$.  Note that $\phi(a+bi)+\phi(c+di)=(a+bx+H)+(c+dx+H)=a+c+(b+d)x+H$ thus $\phi$ preserves addition and so $\phi$ is a homomorphisim.  Note that if $a+bi\in\text{ker}(\phi)$ then $\phi(a+bi)=H$ thus $a+bx\in H$ witch is only true if $a=0$ and $b=0$ thus $\text{ker}(\phi)=0$ and we know that $\phi$ is one-to-one and thus $\phi$ is a isomorphisim and thus $Z_3[i]$ is ring isomorphic to the field $Z_3[x]/<x^2 + 1>$.
\end{exercise}

\begin{exercise}{15.14}
Let $Z[\sqrt{2}]=\{a+b\sqrt{2}\mid a,b\in Z\}$ and
$$H=\biggr\{ \begin{bmatrix}
a & 2b \\ b & a
\end{bmatrix} \biggr|a,b\in Z\biggr\} $$
Show that $Z[\sqrt{2}]$ and $H$ are isomorphic rings.\\
Define $\phi:Z[\sqrt{2}]\rightarrow H$ as $\phi(a+b\sqrt{2})=\begin{bmatrix}
a & 2b \\ b & a
\end{bmatrix}$.  Note that $\phi((a+b\sqrt{2})(c+d\sqrt{2}))=\phi(ac+2bd+(ad+bc)\sqrt{2})$ and $\phi(a+b\sqrt{2})\phi(c+d\sqrt{2})=\begin{bmatrix}
a & 2b \\ b & a
\end{bmatrix}\begin{bmatrix}
c & 2d \\ d & c
\end{bmatrix}=\begin{bmatrix}
ac+2bd & 2(ad+bc) \\ ad+bc & ac+2bd
\end{bmatrix}=\phi(ac+2bd+(ad+bc)\sqrt{2})$ thus $\phi$ is multiplication preserving.  Note that $\phi((a+b\sqrt{2})+(c+d\sqrt{2}))=\phi((a+c)+(b+d)\sqrt{2})=\begin{bmatrix}
a+c & 2(b+d) \\ b+d & a+c
\end{bmatrix}=\begin{bmatrix}
a & 2b \\ b & a
\end{bmatrix}+\begin{bmatrix}
c & 2d \\ d & c
\end{bmatrix}=
\phi(a+b\sqrt{2})+\phi(c+d\sqrt{2})$ and thus $\phi$ is addition preserving.  Note that $\phi$ is a homomorphisim.  By inspection it is clear that $\phi$ is onto and also that $\phi$ is one-to-one, thus $\phi$ is a isomorphisim and so $Z[\sqrt{2}]$ and $H$ are isomorphic rings.
\end{exercise}

\begin{exercise}{15.16}
Let $R=\biggr\{\begin{bmatrix}
a & b \\ 0 & c
\end{bmatrix}
\biggr| a,b,c\in Z\biggr\}$. Prove or disprove that the mapping $\begin{bmatrix}
a & b \\ 0 & c
\end{bmatrix}\rightarrow a$ is a ring homomorphisim.\\
Note that $\phi(\begin{bmatrix}
a & b \\ 0 & c
\end{bmatrix}\begin{bmatrix}
d & e \\ 0 & f
\end{bmatrix})=\phi(\begin{bmatrix}
ad & ? \\ 0 & ?
\end{bmatrix})=ad=\phi(\begin{bmatrix}
a & b \\ 0 & c
\end{bmatrix})\phi(\begin{bmatrix}
d & e \\ 0 & f
\end{bmatrix})$ where $?$ represents something that is not solved here, thus $\phi$ is multiplication preserving.  Note that $\phi(\begin{bmatrix}
a & b \\ 0 & c
\end{bmatrix}+\begin{bmatrix}
d & e \\ 0 & f
\end{bmatrix})=\phi(\begin{bmatrix}
a+d & b+e \\ 0 & c+f
\end{bmatrix})=a+d=\phi(\begin{bmatrix}
a & b \\ 0 & c
\end{bmatrix})+\phi(\begin{bmatrix}
d & e \\ 0 & f
\end{bmatrix})$ thus $\phi$ is addition preserving and thus a homomorphisim.
\end{exercise}

\begin{exercise}{15.20}
Recall that a ring element $a$ is called an idempotent if $a^2 = a$. Prove that a ring homomorphism carries an idempotent to an idempotent.\\
Suppose $\phi:A\rightarrow B$ where $A,B$ are rings and $\phi$ is a homomorphisim.  Suppose $a\in A$ is a idempotent.  Note that $\phi(a)^2=\phi(a^2)=\phi(a)$ thus $\phi(a)$ is a idempotent.  We conclude that a ring homomorphism carries an idempotent to an idempotent.
\end{exercise}

\begin{exercise}{15.32}
Let $n$ be an integer with decimal representation $a_ka_{k-1}\cdots a_1a_0$.  Prove that $n$ is divisible by 11 iff $a_0-a_1+a_2-\cdots (-1)^ka_k$ is devisable by 11.\\
Let $\phi :Z\rightarrow Z_{11}$ be the natural homomorphisim $\phi(a)=a\text{ mod }11$.  Note that $11\mid a$ iff $\phi(a)=0$.  Note that $\phi(\sum_{i=0}^{k} a_k*10^k)=\sum_{i=0}^{k} a_k*\phi(10)^k=\sum_{i=0}^{k} a_k*\phi(-1)^k=\phi(\sum_{i=0}^{k} a_k*(-1)^k)$ thus $11\mid \sum_{i=0}^{k} a_k*10^k$ iff $\phi(\sum_{i=0}^{k} a_k*(-1)^k)=0\longleftrightarrow 11\mid \sum_{i=0}^{k} a_k*(-1)^k$.
\end{exercise}

\begin{exercise}{15.36}
Let $n$ be an integer with decimal representation $a_ka_{k-1}\cdots a_1a_0$.  Prove that $n$ is divisible by 4 iff $a_1a_0$ is devisable by 4.\\
Let $\phi :Z\rightarrow Z_{4}$ be the natural homomorphisim $\phi(a)=a\text{ mod }4$.  Note that $4\mid a$ iff $\phi(a)=0$.  Note that $\phi(\sum_{i=0}^{k} a_k*10^k)=\phi((\sum_{i=2}^{k} a_k*10^k)+a_1a_0)=\phi(\sum_{i=2}^{k} a_k*10^k)+\phi(a_1a_0)=\sum_{i=2}^{k} (\phi(a_k)*\phi(10)^{k-2}*\phi(100))+\phi(a_1a_0)=\sum_{i=2}^{k} (\phi(a_k)*\phi(10)^{k-2}*0)+\phi(a_1a_0)=\phi(a_1a_0)$.  Note that $4\mid \sum_{i=0}^{k} a_k*10^k$ iff $\phi(a_1a_0)=0\longleftrightarrow 4\mid a_1a_0$.
\end{exercise}

\end{document}