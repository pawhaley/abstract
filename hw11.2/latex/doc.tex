\documentclass[12pt]{article}

\usepackage{amssymb,amsmath,amsthm}
\usepackage[top=1in, bottom=1in, left=1.25in, right=1.25in]{geometry}
\usepackage{fancyhdr}
\usepackage{enumerate}
\usepackage[bw,framed,numbered]{mcode}
\usepackage{graphicx}

% Comment the following line to use TeX's default font of Computer Modern.
\usepackage{times,txfonts}

\newtheoremstyle{homework}% name of the style to be used
  {18pt}% measure of space to leave above the theorem. E.g.: 3pt
  {12pt}% measure of space to leave below the theorem. E.g.: 3pt
  {}% name of font to use in the body of the theorem
  {}% measure of space to indent
  {\bfseries}% name of head font
  {:}% punctuation between head and body
  {2ex}% space after theorem head; " " = normal interword space
  {}% Manually specify head
\theoremstyle{homework} 

% Set up an Exercise environment and a Solution label.
\newtheorem*{exercisecore}{Exercise \@currentlabel}
\newenvironment{exercise}[1]
{\def\@currentlabel{#1}\exercisecore}
{\endexercisecore}

\newcommand{\localhead}[1]{\par\smallskip\noindent\textbf{#1}\nobreak\\}%
\newcommand\solution{\localhead{Solution:}}

%%%%%%%%%%%%%%%%%%%%%%%%%%%%%%%%%%%%%%%%%%%%%%%%%%%%%%%%%%%%%%%%%%%%%%%%
%
% Stuff for getting the name/document date/title across the header
\makeatletter
\RequirePackage{fancyhdr}
\pagestyle{fancy}
\fancyfoot[C]{\ifnum \value{page} > 1\relax\thepage\fi}
\fancyhead[L]{\ifx\@doclabel\@empty\else\@doclabel\fi}
\fancyhead[C]{\ifx\@docdate\@empty\else\@docdate\fi}
\fancyhead[R]{\ifx\@docauthor\@empty\else\@docauthor\fi}
\headheight 15pt

\def\doclabel#1{\gdef\@doclabel{#1}}
\doclabel{Use {\tt\textbackslash doclabel\{MY LABEL\}}.}
\def\docdate#1{\gdef\@docdate{#1}}
\docdate{Use {\tt\textbackslash docdate\{MY DATE\}}.}
\def\docauthor#1{\gdef\@docauthor{#1}}
\docauthor{Use {\tt\textbackslash docauthor\{MY NAME\}}.}
\makeatother

% Shortcuts for blackboard bold number sets (reals, integers, etc.)
\newcommand{\Reals}{\ensuremath{\mathbb R}}
\newcommand{\Nats}{\ensuremath{\mathbb N}}
\newcommand{\Ints}{\ensuremath{\mathbb Z}}
\newcommand{\Rats}{\ensuremath{\mathbb Q}}
\newcommand{\Cplx}{\ensuremath{\mathbb C}}
\newcommand{\Aut}{\ensuremath{\text{Aut}}}
%% Some equivalents that some people may prefer.
\let\RR\Reals
\let\NN\Nats
\let\II\Ints
\let\CC\Cplx

%%%%%%%%%%%%%%%%%%%%%%%%%%%%%%%%%%%%%%%%%%%%%%%%%%%%%%%%%%%%%%%%%%%%%%%%%%%%%%%%%%%%%%%
%%%%%%%%%%%%%%%%%%%%%%%%%%%%%%%%%%%%%%%%%%%%%%%%%%%%%%%%%%%%%%%%%%%%%%%%%%%%%%%%%%%%%%%
% 
% The main document start here.

% The following commands set up the material that appears in the header.

%%%%%%%%%%%%%%%%%%%%%%%%%%%%%%%%%%%%%%%%%%%%%%%%%%%%%%%%%%%%%%%%%%%%%%%%%%%%%%%%%%%%%%%
%%%%%%%%%%%%%%%%%%%%%%%%%%%%%%%%%%%%%%%%%%%%%%%%%%%%%%%%%%%%%%%%%%%%%%%%%%%%%%%%%%%%%%%
% 
% The main document start here.

% The following commands set up the material that appears in the header.
\doclabel{Abstract hw 11-2}
\docauthor{Parker Whaley}
\docdate{March 31 2017}
%\lstinputlisting{}
\newcommand{\vv}{\mathbf{v}}
\begin{document}
\begin{exercise}{10.31}
Suppose that $\phi$ is a homomorphism from $U(30)$ to $U(30)$ and that $\ker \phi = \{1, 11\}$. If $\phi(7) = 7$, find all elements of $U(30)$ that map to $7$.\\
The elements that map onto 7 are precisely $a\ker\phi$ where $a\in \phi^{-1}(7)$ thus these elements are $\{7,17\}$.
\end{exercise}

\begin{exercise}{10.35}
Prove that the mapping $\phi: Z \bigoplus Z \rightarrow Z$ given by $(a, b) \rightarrow a - b$ is a homomorphism. What is the kernel of $\phi$? Describe the set $\phi^{-1} (3)$ (that is, all elements that map to 3).\\
Clearly $\phi$ is well defined and thus we need only show that operation are preserved.  Note that $\phi((a,b))\phi((c,d))=a-b+c-d=(a+c)-(b+d)=\phi(((a+c),(b+d)))=\phi((a,b)+(c,d))$, and thus $\phi$ is a homomorphism.\\
Note that $(a,b)\in\ker\phi$ iff $a-b=0$ in other words where $a=b$ thus $\ker\phi =\{ (a,a)\mid a\in\mathbb{Z}\}$.\\
Note that $(3,0)\in\phi^{-1}(3)$ thus $\phi^{-1}(3)=(3,0)\ker\phi=\{ (a+3,a)\mid a\in\mathbb{Z}\}$.
\end{exercise}

\begin{exercise}{10.40}
For each pair of positive integers $m$ and $n$, we can define a homomorphism from $Z$ to $Z_m \bigoplus Z_n$ by $x \rightarrow (x \mod m, x \mod n)$. What is the kernel when $(m, n) = (3, 4)$? What is the kernel when $(m, n) = (6, 4)$? Generalize.\\
Note that $x\in\ker\phi$ iff $\phi(x)=(0,0)$.  Thus $\ker\phi=\{x\in Z\mid \phi(x)=(0,0)\}$ or in our case $\ker\phi=\{x\in Z\mid (m\mid x)\wedge(n\mid x) \}$ or $\ker\phi=\{x\text{ lcm}(m,n)\mid x\in Z\}$.  If $(m, n) = (3, 4)$ then $\ker\phi=\{12x\mid x\in Z\}$.  If $(m, n) = (6, 4)$ then $\ker\phi=\{12x\mid x\in Z\}$.
\end{exercise}

\begin{exercise}{10.43}
Let $\phi(d)$ denote the Euler phi function of $d$ (see page 85). Show that the number of homomorphisms from $Z_n$ to $Z_k$ is $\sum\phi(d)$, where the sum runs over all common divisors $d$ of $n$ and $k$. [It follows from number theory that this sum is actually $\text{gcd}(n, k)$.]\\
First let's break up all of the homomorphisms by the size of the image.  Note that the size of the image call it $d$ must divide $n$ since the homomorphism associated with this image divides the group $Z_n$ into $d$ chunks of equal size.  Note that the size of the image call it $d$ must divide $m$ since $\psi(Z_n)$ is a subgroup of size $d$ in $Z_m$.\\
How many homomorphisms have a size of there image equal to $d$?  Well as discussed above if $d\nmid m$ or $d\nmid n$ then there are no homomorphisms associated with it.  However if $d\mid n$ and $d\mid m$ then we will have homomorphisms associated with it and these homomorphisms map onto $<m/d>$ the only subgroup of $Z_m$ with $d$ elements.  If we know where the generator $1$ in $Z_n$ gets mapped to in $<m/d>$ we know were every item gets mapped to.  Noting that 1 must get mapped to a generator we know that there are exactly as many homomorphisms onto $<m/d>$ as $<m/d>$ has generators.  Note that $<m/d>$ has $\phi(|<m/d>|)=\phi(d)$ generators.\\
Now simply add up the number of homomorphisms associated with any $d$ value and we have the total number of homomorphisms.  The result of this sum is exactly the sum described in the question.
\end{exercise}

\begin{exercise}{10.48}
Suppose that $Z_{10}$ and $Z_{15}$ are both homomorphic images of a finite group $G$. What can be said about $|G|$? Generalize.\\
We know that $|Z_{10}|\mid |G|$ and that $|Z_{15}|\mid |G|$ thus $\text{lcm } (10,15)\mid |G|$.  In general if groups $a_1\cdots a_n$ are homomorphic images of a finite group $G$ then $\text{lcm } (|a_1|\cdots |a_n|)\mid |G|$.
\end{exercise}

\begin{exercise}{10.59}
Suppose that $H$ and $K$ are distinct subgroups of $G$ of index $2$. Prove that $H \cap K$ is a normal subgroup of $G$ of index $4$ and that $G/(H \cap K)$ is not cyclic.\\
Note that $H\not\subset K$ and $K\not\subset H$.  Let $a\in H\cap K^c$ and let $b\in K\cap H^c$.  Suppose $H$ is not normal in $G$.  In this case there exists $a\in G$ and $h\in H$ such that $aha^{-1}\not\in H$.  Note that $a\not\in H$, since $H$ has closure.  Note that $aha^{-1}\not\in H$ implies that $aha^{-1}\in aH$ since $H$ is index 2.  Note that $ha^{-1}\in H$ thus $(ha^{-1})^{-1}\in H$ thus $ah^{-1}\in H$ however $ah^{-1}\in aH$, a contradiction.\\
Note that $H$ and $K$ are normal subgroup of $G$.  Note that $H\cap K=J$ is a normal subgroup of $G$.  Let $a\in H$ and $a\not\in K$, note that $a\not\in J$ and thus $aJ$ is a separate coset from $J$.  Let $b\not\in H$ and $b\in K$, note that $b\not\in J$ and since $b\not\in H=aH$, $b\not\in aJ$, thus $bJ$ is a separate coset from $aJ$ and $J$.  Let $c\not\in H$ and $c\not\in K$, note that $c\not\in J$ and since $c\not\in H=aH$, $c\not\in aJ$, and since $c\not\in K=bK$, $c\not\in bJ$, thus $cJ$ is a separate coset from $bJ$, $aJ$ and $J$.  Thus $J$ is at least of index 4.\\
Note that if $d\in G$, $d$ will fall in one of these 4 cosets.  I will not prove all 4 cases but I will prove one case as example, Suppose $d\not\in H$ and $d\in K$.  Thus $d\in bH$ and $d\in K=bH$, so $d\in bJ$.  Note that we now know that $J$ is exactly of index 4.\\
Note that $G/(H \cap K)=\{J,aJ,bJ,cJ\}$.  Suppose $G/(H \cap K)$ is cyclic.  Note that it must have $2$ generators.  Note that $J$ is not a generator since it is identity and thus $aJ$ or $bJ$ will be a generator.  We can break symmetry at this point between $K$ and $H$ and say WLoG $aJ$ is a generator.  Note that $a^2\in H$ (since $a\in H$) and that $a^2\in K$ since if $a^2\not\in K$ we would know that $a^2\in aJ$ and thus $(aJ)^2=aJ$ a impossibility, thus $(aJ)^2=J=e$.  We have reached a contradiction since $aJ$ is a generator, thus we negate our supposition and conclude that $G/(H \cap K)$ is not cyclic.
\end{exercise}


\end{document}