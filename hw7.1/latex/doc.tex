\documentclass[12pt]{article}

\usepackage{amssymb,amsmath,amsthm}
\usepackage[top=1in, bottom=1in, left=1.25in, right=1.25in]{geometry}
\usepackage{fancyhdr}
\usepackage{enumerate}
\usepackage[bw,framed,numbered]{mcode}
\usepackage{graphicx}

% Comment the following line to use TeX's default font of Computer Modern.
\usepackage{times,txfonts}

\newtheoremstyle{homework}% name of the style to be used
  {18pt}% measure of space to leave above the theorem. E.g.: 3pt
  {12pt}% measure of space to leave below the theorem. E.g.: 3pt
  {}% name of font to use in the body of the theorem
  {}% measure of space to indent
  {\bfseries}% name of head font
  {:}% punctuation between head and body
  {2ex}% space after theorem head; " " = normal interword space
  {}% Manually specify head
\theoremstyle{homework} 

% Set up an Exercise environment and a Solution label.
\newtheorem*{exercisecore}{Exercise \@currentlabel}
\newenvironment{exercise}[1]
{\def\@currentlabel{#1}\exercisecore}
{\endexercisecore}

\newcommand{\localhead}[1]{\par\smallskip\noindent\textbf{#1}\nobreak\\}%
\newcommand\solution{\localhead{Solution:}}

%%%%%%%%%%%%%%%%%%%%%%%%%%%%%%%%%%%%%%%%%%%%%%%%%%%%%%%%%%%%%%%%%%%%%%%%
%
% Stuff for getting the name/document date/title across the header
\makeatletter
\RequirePackage{fancyhdr}
\pagestyle{fancy}
\fancyfoot[C]{\ifnum \value{page} > 1\relax\thepage\fi}
\fancyhead[L]{\ifx\@doclabel\@empty\else\@doclabel\fi}
\fancyhead[C]{\ifx\@docdate\@empty\else\@docdate\fi}
\fancyhead[R]{\ifx\@docauthor\@empty\else\@docauthor\fi}
\headheight 15pt

\def\doclabel#1{\gdef\@doclabel{#1}}
\doclabel{Use {\tt\textbackslash doclabel\{MY LABEL\}}.}
\def\docdate#1{\gdef\@docdate{#1}}
\docdate{Use {\tt\textbackslash docdate\{MY DATE\}}.}
\def\docauthor#1{\gdef\@docauthor{#1}}
\docauthor{Use {\tt\textbackslash docauthor\{MY NAME\}}.}
\makeatother

% Shortcuts for blackboard bold number sets (reals, integers, etc.)
\newcommand{\Reals}{\ensuremath{\mathbb R}}
\newcommand{\Nats}{\ensuremath{\mathbb N}}
\newcommand{\Ints}{\ensuremath{\mathbb Z}}
\newcommand{\Rats}{\ensuremath{\mathbb Q}}
\newcommand{\Cplx}{\ensuremath{\mathbb C}}
\newcommand{\Aut}{\ensuremath{\text{Aut}}}
%% Some equivalents that some people may prefer.
\let\RR\Reals
\let\NN\Nats
\let\II\Ints
\let\CC\Cplx

%%%%%%%%%%%%%%%%%%%%%%%%%%%%%%%%%%%%%%%%%%%%%%%%%%%%%%%%%%%%%%%%%%%%%%%%%%%%%%%%%%%%%%%
%%%%%%%%%%%%%%%%%%%%%%%%%%%%%%%%%%%%%%%%%%%%%%%%%%%%%%%%%%%%%%%%%%%%%%%%%%%%%%%%%%%%%%%
% 
% The main document start here.

% The following commands set up the material that appears in the header.

%%%%%%%%%%%%%%%%%%%%%%%%%%%%%%%%%%%%%%%%%%%%%%%%%%%%%%%%%%%%%%%%%%%%%%%%%%%%%%%%%%%%%%%
%%%%%%%%%%%%%%%%%%%%%%%%%%%%%%%%%%%%%%%%%%%%%%%%%%%%%%%%%%%%%%%%%%%%%%%%%%%%%%%%%%%%%%%
% 
% The main document start here.

% The following commands set up the material that appears in the header.
\doclabel{Abstract hw 7-1}
\docauthor{Parker Whaley}
\docdate{Feb 23, 2017}

\newcommand{\vv}{\mathbf{v}}
\begin{document}
\begin{exercise}{6.12}
Find two groups $G$ and $H$ such that $G \not\approx H$, but $\Aut(G) \approx \Aut(H)$.\\
Let $G=Z_4$ and $H=Z_6$.  Note that they have different numbers of elements thus $G \not\approx H$.  Note that they each have two auto-morphisims, the identity and the exchange of generators, and all groups with two elements are isomorphic, thus $\Aut(G) \approx \Aut(H)$.
\end{exercise}

\begin{exercise}{6.14}
Find $\Aut(Z_6 )$.\\
As descused in the previous question there are two elements to this group,
$$\text{The identity }\begin{pmatrix}
0&1&2&3&4&5\\
0&1&2&3&4&5
\end{pmatrix}$$
$$\text{The exchange of generators }\begin{pmatrix}
0&1&2&3&4&5\\
0&5&4&3&2&1
\end{pmatrix}$$
\end{exercise}

\begin{exercise}{6.20}
Show that $Z$ has infinitely many subgroups isomorphic to $Z$.\\
Choose $n\in \mathbb{N}$.  Define $H\subseteq Z$ as $H=\{k\in Z : n\mid k\}$.\\
Choose $a,b\in H$.  Note that $a=nc$ and $b=nd$ for some integers $c,d$.  Note that $ab^{-1}=nc-nd=n(c-d)\in H$.  Thus $H$ is a group, and so $H$ is a subgroup of $Z$.\\
Define $\phi(x)=nx$.  Note that $\phi: Z\rightarrow H$ bijectively.  Note that $\phi(ab)=\phi(a+b)=n(a+b)=na+nb=\phi(a)+\phi(b)=\phi(a)\phi(b)$, thus $H\approx Z$.\\
Since $H$ is unique depending on our choice of $n$ and there are a infinite number of possible $n$'s we can say that $Z$ has infinitely many subgroups isomorphic to $Z$.
\end{exercise}

\begin{exercise}{6.23}
Give an example of a cyclic group of smallest order that contains a subgroup isomorphic to $Z_{12}$ and a subgroup isomorphic to $Z_{20}$.  No need to prove anything, but explain your reasoning.\\
Essentially we are looking for a $Z_n$ witch contains a element of order $12$ and a element of order $20$.  This must mean $12\mid n$ and $20\mid n$, the smallest $n$ with this property is $60$.  Note that in $Z_{60}$, $|<5>|=12$ and $|<3>|=20$, thus, do to all cyclic groups of the same order being isomorphic, $<5>\approx Z_{12}$ and $<3>\approx Z_{20}$.  I have demonstrated that $Z_{60}$ is the smallest that could have this property and that it does have this property.
\end{exercise}

\begin{exercise}{6.24}
Suppose that $\phi : Z_{20} \rightarrow Z_{20}$ is an automorphism and $\phi(5) = 5$. What are the possibilities for $\phi(x)$?\\
To require that $\phi$ is a automorphism is exactly to require that it map at least one generator to another generator.  Note that $5=\phi(5)=\phi(1^5)=\phi(1)^5$, noting that $\phi(1)$ is a generator it is only left to check witch generators have this property.
\newpage
\lstinputlisting{../octave/d1.txt}
We now see that the automorphisms that work are $\phi(1)=1$, $\phi(1)=9$, $\phi(1)=13$, $\phi(1)=17$.  Note that defining where the generator goes defines a entire automorphism and thus what I have given are the complete descriptions of the 4 automorphisims with the property $\phi(5) = 5$.
\end{exercise}

\begin{exercise}{6.26}
Prove that the mapping from $U(16)$ to itself given by $x \rightarrow x^3$ is an automorphism. What about $x \rightarrow x^5$ and $x \rightarrow x^7$? Generalize.\\
Take $\phi_n: x\rightarrow x^n$.  Note that $U(16)$ is a abelian group thus $\phi_n(ab)=(ab)^n=a^nb^n=\phi_n(a)\phi_n(b)$.  Now all we need to show isomorphisim is bijectivity.  Since this is a finite group it is sufficient to show that $\phi_n$ is onto.\\
\lstinputlisting{../octave/d2.txt}
As demonstrated above $\phi_3$, $\phi_5$, $\phi_7$ work, however note that $\phi_2$ will not work as a automorphism.
\end{exercise}







\end{document}