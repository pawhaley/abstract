\documentclass[12pt]{article}

\usepackage{amssymb,amsmath,amsthm}
\usepackage[top=1in, bottom=1in, left=1.25in, right=1.25in]{geometry}
\usepackage{fancyhdr}
\usepackage{enumerate}
\usepackage[bw,framed,numbered]{mcode}
\usepackage{graphicx}

% Comment the following line to use TeX's default font of Computer Modern.
\usepackage{times,txfonts}

\newtheoremstyle{homework}% name of the style to be used
  {18pt}% measure of space to leave above the theorem. E.g.: 3pt
  {12pt}% measure of space to leave below the theorem. E.g.: 3pt
  {}% name of font to use in the body of the theorem
  {}% measure of space to indent
  {\bfseries}% name of head font
  {:}% punctuation between head and body
  {2ex}% space after theorem head; " " = normal interword space
  {}% Manually specify head
\theoremstyle{homework} 

% Set up an Exercise environment and a Solution label.
\newtheorem*{exercisecore}{Exercise \@currentlabel}
\newenvironment{exercise}[1]
{\def\@currentlabel{#1}\exercisecore}
{\endexercisecore}

\newcommand{\localhead}[1]{\par\smallskip\noindent\textbf{#1}\nobreak\\}%
\newcommand\solution{\localhead{Solution:}}

%%%%%%%%%%%%%%%%%%%%%%%%%%%%%%%%%%%%%%%%%%%%%%%%%%%%%%%%%%%%%%%%%%%%%%%%
%
% Stuff for getting the name/document date/title across the header
\makeatletter
\RequirePackage{fancyhdr}
\pagestyle{fancy}
\fancyfoot[C]{\ifnum \value{page} > 1\relax\thepage\fi}
\fancyhead[L]{\ifx\@doclabel\@empty\else\@doclabel\fi}
\fancyhead[C]{\ifx\@docdate\@empty\else\@docdate\fi}
\fancyhead[R]{\ifx\@docauthor\@empty\else\@docauthor\fi}
\headheight 15pt

\def\doclabel#1{\gdef\@doclabel{#1}}
\doclabel{Use {\tt\textbackslash doclabel\{MY LABEL\}}.}
\def\docdate#1{\gdef\@docdate{#1}}
\docdate{Use {\tt\textbackslash docdate\{MY DATE\}}.}
\def\docauthor#1{\gdef\@docauthor{#1}}
\docauthor{Use {\tt\textbackslash docauthor\{MY NAME\}}.}
\makeatother

% Shortcuts for blackboard bold number sets (reals, integers, etc.)
\newcommand{\Reals}{\ensuremath{\mathbb R}}
\newcommand{\Nats}{\ensuremath{\mathbb N}}
\newcommand{\Ints}{\ensuremath{\mathbb Z}}
\newcommand{\Rats}{\ensuremath{\mathbb Q}}
\newcommand{\Cplx}{\ensuremath{\mathbb C}}
\newcommand{\Aut}{\ensuremath{\text{Aut}}}
%% Some equivalents that some people may prefer.
\let\RR\Reals
\let\NN\Nats
\let\II\Ints
\let\CC\Cplx

%%%%%%%%%%%%%%%%%%%%%%%%%%%%%%%%%%%%%%%%%%%%%%%%%%%%%%%%%%%%%%%%%%%%%%%%%%%%%%%%%%%%%%%
%%%%%%%%%%%%%%%%%%%%%%%%%%%%%%%%%%%%%%%%%%%%%%%%%%%%%%%%%%%%%%%%%%%%%%%%%%%%%%%%%%%%%%%
% 
% The main document start here.

% The following commands set up the material that appears in the header.

%%%%%%%%%%%%%%%%%%%%%%%%%%%%%%%%%%%%%%%%%%%%%%%%%%%%%%%%%%%%%%%%%%%%%%%%%%%%%%%%%%%%%%%
%%%%%%%%%%%%%%%%%%%%%%%%%%%%%%%%%%%%%%%%%%%%%%%%%%%%%%%%%%%%%%%%%%%%%%%%%%%%%%%%%%%%%%%
% 
% The main document start here.

% The following commands set up the material that appears in the header.
\doclabel{Abstract hw 9}
\docauthor{Parker Whaley}
\docdate{March 10 2017}
%\lstinputlisting{}
\newcommand{\vv}{\mathbf{v}}
\begin{document}
\begin{exercise}{8.26}
The group $S_3 \bigoplus Z_2$ is isomorphic to one of the following groups:  $Z_{12}$ , $Z_6 \bigoplus Z_2$ , $A_4$ , $D_6$ . Determine which one by elimination.\\
It cannot be isomorphic to $Z_{12}$ since $Z_{12}$ is cyclic and $S_3$ is not.  Since $S_3$ is not abelian we know that it is not isomorphic to $Z_6 \bigoplus Z_2$.  Note that $A_4$ has 8 elements of order 3 and $S_3 \bigoplus Z_2$ has 2 thus they are not isomorphic.  By elimination $S_3 \bigoplus Z_2\\approx D_6$.
\end{exercise}

\begin{exercise}{8.32}
What is the order of the largest cyclic subgroup of $Z_6 \bigoplus Z_{10} \bigoplus Z_{15}$?  What is the order of the largest cyclic subgroup of $Z_{n_1} \bigoplus Z_{n_2} \bigoplus \cdots \bigoplus Z_{n_k}$?\\
Noting that all of these $Z_n$ are cyclic we see that the largest cyclic sub group will be generated by $(g_1,g_2,\cdots,g_k)$ where $g_i$ is the generator for $i$th group in the product.  Note that $|(g_1,g_2,\cdots,g_k)|=\text{lcm}(|g_1 |, |g_2 |, \cdots , |g_n |)=\text{lcm}(n_1, |n_2 |, \cdots , |n_3 |)$.  Thus for $Z_6 \bigoplus Z_{10} \bigoplus Z_{15}$ the largest cyclic sub group has order $\text{lcm}(2*3,2*5,3*5)=2*3*5=30$.
\end{exercise}

\begin{exercise}{8.55}
How many isomorphisms are there from $Z_{12}$ to $Z_4 \bigoplus Z_3$ ?\\
Note that $Z_4 \bigoplus Z_3$ is cyclic and has twelve elements thus there is at least one isomorphisim.
Deciding where to map a generator uniquely defines a isomorphisim, thus we need only examine what we map to the generator $(1,1)\in Z_4\bigoplus Z_3$.  We know we can only map generators to that element and the generators in $Z_{12}$ are 1,5,7,11.  Thus we conclude there are exactly 4 isomorphic
\end{exercise}

\begin{exercise}{8.58}
Prove that $Z_5 \bigoplus Z_5$ has exactly six subgroups of order 5.\\
Note that
$$<(1,0)>=\{(1,0),(2,0),(3,0),(4,0),(0,0)\}$$
$$<(1,1)>=\{(1,1),(2,2),(3,3),(4,4),(0,0)\}$$
$$<(1,2)>=\{(1,2),(2,4),(3,1),(4,3),(0,0)\}$$
$$<(1,3)>=\{(1,3),(2,1),(3,4),(4,2),(0,0)\}$$
$$<(1,4)>=\{(1,4),(2,3),(3,2),(4,1),(0,0)\}$$
$$<(0,1)>=\{(0,1),(0,2),(0,3),(0,4),(0,0)\}$$
thus there are at least 6 sub groups of order 5.\\
Suppose $Z_5 \bigoplus Z_5$ had 7 subgroups of order 5.  Note that each of these sub groups must be cyclic, since 5 is prime.  Note that each of the non identity elements are generators of there sub group since 5 is prime.  Thus we see that each sub group has the identity and 4 elements in no other sub group thus there are at least $1+5*7=36$ elements, a contradiction since the group only has 25 elements, thus there are exactly 6 sub groups of order 5.
\end{exercise}

\begin{exercise}{8.66}
Express $U(165)$ as an external direct product of cyclic groups of the form $Z_n$.\\
Note that $U(165)=U(5*3*11)\approx U(3)\bigoplus U(5)\bigoplus U(11)\approx Z_2\bigoplus Z_4\bigoplus Z_{10}$.  (see pg 167)
\end{exercise}

\begin{exercise}{8.67}
Express $U(165)$ as an external direct product of U-groups in four different ways.\\
$$U(165)=U(3*55)\approx U(3)\bigoplus U(55)$$
$$U(165)=U(5*33)\approx U(5)\bigoplus U(33)$$
$$U(165)=U(11*15)\approx U(11)\bigoplus U(15)$$
$$U(165)=U(5*3*11)\approx U(3)\bigoplus U(5)\bigoplus U(11)$$
\end{exercise}

\begin{exercise}{8.68}
Without doing any calculations in $\text{Aut}(Z_{20})$, determine how many elements of $\text{Aut}(Z_{20})$ have order $4$. How many have order $2$?\\
Note that $\text{Aut}(Z_{20})\approx U(20)\approx U(2^2)\bigoplus U(5)\approx Z_2\bigoplus Z_{4}$.  Note that $Z_2\bigoplus Z_{4}$ has 4 elements of order 4 and 3 elements of order 2, thus $\text{Aut}(Z_{20})$ has 4 elements of order 4 and 3 elements of order 2.
\end{exercise}

\begin{exercise}{8.78}
Find a subgroup of order 6 in $U(700)$. (see appendix)\\
From the table below $<51>$ is a subgroup of order 6.
$<51>=\{51,501,351,401,151,1\}$

\end{exercise}


\newpage
table
\lstinputlisting{../octave/d1.txt}



\end{document}